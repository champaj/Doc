%% Generated by Sphinx.
\def\sphinxdocclass{report}
\documentclass[a4paper,10pt,english]{report}
\ifdefined\pdfpxdimen
   \let\sphinxpxdimen\pdfpxdimen\else\newdimen\sphinxpxdimen
\fi \sphinxpxdimen=.75bp\relax

\PassOptionsToPackage{warn}{textcomp}
\usepackage[utf8]{inputenc}
\ifdefined\DeclareUnicodeCharacter
% support both utf8 and utf8x syntaxes
\edef\sphinxdqmaybe{\ifdefined\DeclareUnicodeCharacterAsOptional\string"\fi}
  \DeclareUnicodeCharacter{\sphinxdqmaybe00A0}{\nobreakspace}
  \DeclareUnicodeCharacter{\sphinxdqmaybe2500}{\sphinxunichar{2500}}
  \DeclareUnicodeCharacter{\sphinxdqmaybe2502}{\sphinxunichar{2502}}
  \DeclareUnicodeCharacter{\sphinxdqmaybe2514}{\sphinxunichar{2514}}
  \DeclareUnicodeCharacter{\sphinxdqmaybe251C}{\sphinxunichar{251C}}
  \DeclareUnicodeCharacter{\sphinxdqmaybe2572}{\textbackslash}
\fi
\usepackage{cmap}
\usepackage[T1]{fontenc}
\usepackage{amsmath,amssymb,amstext}
\usepackage{babel}
\usepackage{amsmath,amsfonts,amssymb,amsthm}
\usepackage{fncychap}
\usepackage[,numfigreset=1,mathnumfig]{sphinx}
\sphinxsetup{hmargin={0.7in,0.7in}, vmargin={1in,1in},         verbatimwithframe=true,         TitleColor={rgb}{0,0,0},         HeaderFamily=\rmfamily\bfseries,         InnerLinkColor={rgb}{0,0,1},         OuterLinkColor={rgb}{0,0,1}}
\fvset{fontsize=\small}
\usepackage{geometry}

% Include hyperref last.
\usepackage{hyperref}
% Fix anchor placement for figures with captions.
\usepackage{hypcap}% it must be loaded after hyperref.
% Set up styles of URL: it should be placed after hyperref.
\urlstyle{same}
\addto\captionsenglish{\renewcommand{\contentsname}{Let's look further:}}

\addto\captionsenglish{\renewcommand{\figurename}{Fig.\@ }}
\makeatletter
\def\fnum@figure{\figurename\thefigure{}}
\makeatother
\addto\captionsenglish{\renewcommand{\tablename}{Table }}
\makeatletter
\def\fnum@table{\tablename\thetable{}}
\makeatother
\addto\captionsenglish{\renewcommand{\literalblockname}{Listing}}

\addto\captionsenglish{\renewcommand{\literalblockcontinuedname}{continued from previous page}}
\addto\captionsenglish{\renewcommand{\literalblockcontinuesname}{continues on next page}}
\addto\captionsenglish{\renewcommand{\sphinxnonalphabeticalgroupname}{Non-alphabetical}}
\addto\captionsenglish{\renewcommand{\sphinxsymbolsname}{Symbols}}
\addto\captionsenglish{\renewcommand{\sphinxnumbersname}{Numbers}}

\addto\extrasenglish{\def\pageautorefname{page}}

\setcounter{tocdepth}{2}


        %%%%%%%%%%%%%%%%%%%% Meher %%%%%%%%%%%%%%%%%%
        %%%add number to subsubsection 2=subsection, 3=subsubsection
        %%% below subsubsection is not good idea.
        \setcounter{secnumdepth}{3}
        %
        %%%% Table of content upto 2=subsection, 3=subsubsection
        \setcounter{tocdepth}{2}

        \usepackage{amsmath,amsfonts,amssymb,amsthm}
        \usepackage{graphicx}

        %%% reduce spaces for Table of contents, figures and tables
        %%% it is used "\addtocontents{toc}{\vskip -1.2cm}" etc. in the document
        \usepackage[notlot,nottoc,notlof]{}

        \usepackage{color}
        \usepackage{transparent}
        \usepackage{eso-pic}
        \usepackage{lipsum}

        \usepackage{footnotebackref} %%link at the footnote to go to the place of footnote in the text

        %% spacing between line
        \usepackage{setspace}
        %%%%\onehalfspacing
        %%%%\doublespacing
        \singlespacing


        %%%%%%%%%%% datetime
        \usepackage{datetime}

        \newdateformat{MonthYearFormat}{%
            \monthname[\THEMONTH], \THEYEAR}


        %% RO, LE will not work for 'oneside' layout.
        %% Change oneside to twoside in document class
        \usepackage{fancyhdr}
        \pagestyle{fancy}
        \fancyhf{}

        %%% Alternating Header for oneside
        \fancyhead[L]{\ifthenelse{\isodd{\value{page}}}{ \small \nouppercase{\leftmark} }{}}
        \fancyhead[R]{\ifthenelse{\isodd{\value{page}}}{}{ \small \nouppercase{\rightmark} }}

        %%% Alternating Header for two side
        %\fancyhead[RO]{\small \nouppercase{\rightmark}}
        %\fancyhead[LE]{\small \nouppercase{\leftmark}}


        %%% page number
        \fancyfoot[CO, CE]{\thepage}

        \renewcommand{\headrulewidth}{0.5pt}
        \renewcommand{\footrulewidth}{0.5pt}

        \RequirePackage{tocbibind} %%% comment this to remove page number for following
        \addto\captionsenglish{\renewcommand{\contentsname}{Table of contents}}
        \addto\captionsenglish{\renewcommand{\listfigurename}{List of figures}}
        \addto\captionsenglish{\renewcommand{\listtablename}{List of tables}}
        % \addto\captionsenglish{\renewcommand{\chaptername}{Chapter}}


        %%reduce spacing for itemize
        \usepackage{enumitem}
        \setlist{nosep}

        %%%%%%%%%%% Quote Styles at the top of chapter
        \usepackage{epigraph}
        \setlength{\epigraphwidth}{0.8\columnwidth}
        \newcommand{\chapterquote}[2]{\epigraphhead[60]{\epigraph{\textit{#1}}{\textbf {\textit{--#2}}}}}
        %%%%%%%%%%% Quote for all places except Chapter
        \newcommand{\sectionquote}[2]{{\quote{\textit{``#1''}}{\textbf {\textit{--#2}}}}}
    

\title{Sphinx format for Latex and HTML}
\date{Aug 05, 2019}
\release{}
\author{aj}
\newcommand{\sphinxlogo}{\sphinxincludegraphics{logo.jpg}\par}
\renewcommand{\releasename}{ }
\makeindex
\begin{document}

\pagestyle{empty}

        \pagenumbering{Roman} %%% to avoid page 1 conflict with actual page 1

        \begin{titlepage}
            \centering

            \vspace*{40mm} %%% * is used to give space from top
            \textbf{\Huge {Sphinx format for Latex and HTML}}

            \vspace{0mm}
            \begin{figure}[!h]
                \centering
                \includegraphics[scale=0.3]{logo.jpg}
            \end{figure}

            \vspace{0mm}
            \Large \textbf{{aj}}

            \small Created on : Octorber, 2019

            \vspace*{0mm}
            \small  Last updated : \MonthYearFormat\today


            %% \vfill adds at the bottom
            \vfill
            \small \textit{More documents are freely available at }
        \end{titlepage}

        \clearpage
        \pagenumbering{roman}
        \tableofcontents
        \listoffigures
        \listoftables
        \clearpage
        \pagenumbering{arabic}

        
\pagestyle{plain}
 
\pagestyle{normal}
\phantomsection\label{\detokenize{index::doc}}


\begin{figure}[htbp]
\centering

\noindent\sphinxincludegraphics{{welcome}.jpeg}
\end{figure}

Welcome to \sphinxstylestrong{Monomerce}. Here you can find the path to Monomerce (an E-Commerce platform).
Follow the path it will take you to the most easiest and organised way of online business but you have to register to \sphinxstylestrong{Monomerce} to know the full specifications.

In above image you are seeing some \sphinxstyleemphasis{numeric marks} like \sphinxincludegraphics{{numeric}.jpeg} it might also be in \sphinxstyleemphasis{alphabetical order} to make this convenient for you people. let’s see how?
\begin{enumerate}
\def\theenumi{\arabic{enumi}}
\def\labelenumi{\theenumi .}
\makeatletter\def\p@enumii{\p@enumi \theenumi .}\makeatother
\item {} 
At the right side of this numeric mark there is \sphinxstyleemphasis{company’s logo} which is a Home page buttton it will be in almost every pages if you want to come back at the home page of this website just click over it it will take you to homwe page form anywhere in this website .

\item {} 
Here is a \sphinxstylestrong{Search Field} in which you can search for products for e.g if you will type \sphinxstyleemphasis{rice} then the below view will be appear.

\end{enumerate}

\begin{figure}[htbp]
\centering
\capstart

\noindent\sphinxincludegraphics{{rice}.jpeg}
\caption{Products View}\label{\detokenize{index:id12}}\label{\detokenize{index:id2}}\end{figure}

Here you
\begin{enumerate}
\def\theenumi{\arabic{enumi}}
\def\labelenumi{\theenumi .}
\makeatletter\def\p@enumii{\p@enumi \theenumi .}\makeatother
\item {} 
Enter name of the poduct which you want to see and buy.

\item {} 
Click on product to see the details.

\end{enumerate}

\begin{figure}[htbp]
\centering
\capstart

\noindent\sphinxincludegraphics{{prod}.png}
\caption{Add products to cart}\label{\detokenize{index:id13}}\label{\detokenize{index:id3}}\end{figure}
\begin{description}
\item[{now}] \leavevmode\begin{enumerate}
\def\theenumi{\alph{enumi}}
\def\labelenumi{\theenumi .}
\makeatletter\def\p@enumii{\p@enumi \theenumi .}\makeatother
\item {} 
Here you can select or customise the quantity of that product it will show the \sphinxstyleemphasis{cost} of that product according to your customise quantity.

\item {} 
Click on \sphinxstylestrong{Add To Cart} here to add the product into cart.

\item {} 
Click on heart shape to add the product into your \sphinxstyleemphasis{wish list}.

\item {} 
This one is also a \sphinxstyleemphasis{Add To Cart} Button to add product into cart, Here you are seeing the minimized view of product, it helps you to add many product at once. As you will click on \sphinxstylestrong{Add To Cart} you will see something like this.

\end{enumerate}

\end{description}

\begin{figure}[htbp]
\centering
\capstart

\noindent\sphinxincludegraphics{{checkout}.png}
\caption{Checking out}\label{\detokenize{index:id14}}\label{\detokenize{index:id4}}\end{figure}
\begin{description}
\item[{now}] \leavevmode\begin{enumerate}
\def\theenumi{\alph{enumi}}
\def\labelenumi{\theenumi .}
\makeatletter\def\p@enumii{\p@enumi \theenumi .}\makeatother
\item {} 
Here you are seeing \sphinxstylestrong{- \& +} button by clicking on it you can remove and add the product as per your  wish and it will show you the bill (Total Cost) accordingly.

\item {} 
This is \sphinxstylestrong{Check Out} button, if you want to checkout then click here else you can add more product to your cart, as you will click on checkout you will see below tab.

\end{enumerate}

\end{description}

\begin{figure}[htbp]
\centering
\capstart

\noindent\sphinxincludegraphics{{checkreview}.png}
\caption{Products Review}\label{\detokenize{index:id15}}\label{\detokenize{index:id5}}\end{figure}

now
\begin{enumerate}
\def\theenumi{\alph{enumi}}
\def\labelenumi{\theenumi .}
\makeatletter\def\p@enumii{\p@enumi \theenumi .}\makeatother
\item {} 
Here you are seeing \sphinxstyleemphasis{Number of items} in your cart.

\item {} 
Here you can increase and descrease the quantity accordingly price will change.

\item {} 
Here apply Promo Code, if you have then it will be applied.

\item {} 
Now click on \sphinxstylestrong{Next} it will direct you to \sphinxstyleemphasis{Shipment Details} tab.

\end{enumerate}

\begin{figure}[htbp]
\centering
\capstart

\noindent\sphinxincludegraphics{{shipment}.png}
\caption{Shipping/Billing Address}\label{\detokenize{index:id16}}\label{\detokenize{index:id6}}\end{figure}

Now here you will fill your Shipping and Billing Address.
\begin{quote}
\begin{enumerate}
\def\theenumi{\alph{enumi}}
\def\labelenumi{\theenumi .}
\makeatletter\def\p@enumii{\p@enumi \theenumi .}\makeatother
\item {} 
Enter your Mobile Number here.

\item {} 
Enter your street name to street number.

\item {} 
Here you have to fill Landmark.

\item {} 
Here Pin Code.

\item {} 
State  f. City  g. Country these all will be auto filled as per your entered pin code, If These are not Matching fill it manually.

\end{enumerate}
\begin{enumerate}
\def\theenumi{\alph{enumi}}
\def\labelenumi{\theenumi .}
\makeatletter\def\p@enumii{\p@enumi \theenumi .}\makeatother
\setcounter{enumi}{7}
\item {} 
This is a Check box click on it (Make it checked) if your billing and shipping address is same else make it unchecked, If you will make it check then you have to enter your Billing Address as you entered above for Shipping Address.

\item {} 
Then click on \sphinxstylestrong{NEXT} to pay the bill.

\item {} 
If you have to do some change you can click on \sphinxstylestrong{GO BACK} button, it will take you to previous portal.

\item {} 
Here is one orange color button click on it, if you want to save your address for Later Use.

\item {} 
If you have entered wrong Address or you want to change it click on \sphinxstylestrong{Cancle} button, If you click on \sphinxstyleemphasis{next} you will see the payment options which will look like

\end{enumerate}
\end{quote}

\begin{figure}[htbp]
\centering
\capstart

\noindent\sphinxincludegraphics{{pay}.png}
\caption{Payment Process}\label{\detokenize{index:id17}}\label{\detokenize{index:id7}}\end{figure}

now
\begin{quote}
\begin{enumerate}
\def\theenumi{\alph{enumi}}
\def\labelenumi{\theenumi .}
\makeatletter\def\p@enumii{\p@enumi \theenumi .}\makeatother
\item {} 
Here you are seeing the payment mode like credit Card. Debit Card, Net Banking and UPI.

\item {} 
Click on \sphinxstylestrong{PAY} and it will take you to payment process which is named  c. \sphinxstylestrong{PAY} in our website, and the payment process is same as for any other online shopping payment.

\end{enumerate}
\begin{enumerate}
\def\theenumi{\alph{enumi}}
\def\labelenumi{\theenumi .}
\makeatletter\def\p@enumii{\p@enumi \theenumi .}\makeatother
\setcounter{enumi}{2}
\item {} 
If you know about product click on \sphinxstylestrong{ADD TO CART} it will be added into your cart.

\end{enumerate}
\begin{enumerate}
\def\theenumi{\arabic{enumi}}
\def\labelenumi{\theenumi .}
\makeatletter\def\p@enumii{\p@enumi \theenumi .}\makeatother
\setcounter{enumi}{2}
\item {} 
It is a link to direct you to \sphinxstylestrong{Login}  page.

\item {} 
Click on this \sphinxstylestrong{Register} it will take you to register page.

\item {} 
Click on \sphinxstylestrong{Cart} icon it will ask you for login then you can check your added item in cart section, if you already loged in then it will show your added item.

\item {} 
Here is \sphinxstylestrong{Company Banner} section.

\item {} 
Here you are seeing \sphinxstylestrong{Products Category} .

\end{enumerate}
\end{quote}

As you are scrolling you will see

\begin{figure}[htbp]
\centering
\capstart

\noindent\sphinxincludegraphics{{footer}.jpeg}
\caption{Footer Portion of Your website}\label{\detokenize{index:id18}}\label{\detokenize{index:id8}}\end{figure}

now you are at footer section of home page.
\begin{enumerate}
\def\theenumi{\arabic{enumi}}
\def\labelenumi{\theenumi .}
\makeatletter\def\p@enumii{\p@enumi \theenumi .}\makeatother
\setcounter{enumi}{7}
\item {} 
This is a Button labeled as \sphinxstylestrong{Show More} by clicking on it you will see more products as per your request.

\item {} 
This is designed to \sphinxstylestrong{Track Your Order} , means you can check the list of your ordered item and what is the status of your orders.

\item {} 
If you are not happy with product and you want to \sphinxstylestrong{return} it click here it will take you to orders tab.

\end{enumerate}


\chapter{Orders:}
\label{\detokenize{index:orders}}\label{\detokenize{index:id9}}
\begin{figure}[htbp]
\centering
\capstart

\noindent\sphinxincludegraphics{{order}.png}
\caption{Check Your Orders}\label{\detokenize{index:id19}}\end{figure}

now
\begin{enumerate}
\def\theenumi{\alph{enumi}}
\def\labelenumi{\theenumi .}
\makeatletter\def\p@enumii{\p@enumi \theenumi .}\makeatother
\item {} 
Here you can search your order \sphinxstylestrong{By Status} for eg. created, shipped. delivered etc.

\item {} 
Here you can see your product details.

\item {} 
Click at this \sphinxstylestrong{i} symbol it will explore this product like

\end{enumerate}

\begin{figure}[htbp]
\centering
\capstart

\noindent\sphinxincludegraphics{{cancle}.png}
\caption{Cancel/Return Order}\label{\detokenize{index:id20}}\label{\detokenize{index:id10}}\end{figure}
\begin{description}
\item[{now}] \leavevmode\begin{enumerate}
\def\theenumi{\alph{enumi}}
\def\labelenumi{\theenumi .}
\makeatletter\def\p@enumii{\p@enumi \theenumi .}\makeatother
\item {} 
Click on this checkbox to select your product to cancle or return.

\item {} 
Here if your product is delivered to you then you can click on \sphinxstylestrong{Return} else you can \sphinxstylestrong{Cancle} then it will ask for confirmation click on \sphinxstylestrong{Yes} if you want to \sphinxstyleemphasis{cancle or Return} otherwise click on \sphinxstylestrong{No} .

\item {} 
by clicking here you can check your order status like \sphinxstyleemphasis{Your order is yet to be confirmed}  etc.

\end{enumerate}
\begin{enumerate}
\def\theenumi{\alph{enumi}}
\def\labelenumi{\theenumi .}
\makeatletter\def\p@enumii{\p@enumi \theenumi .}\makeatother
\setcounter{enumi}{3}
\item {} 
This is a button to download \sphinxstylestrong{Invoice} of your ordered products.

\end{enumerate}
\begin{enumerate}
\def\theenumi{\arabic{enumi}}
\def\labelenumi{\theenumi .}
\makeatletter\def\p@enumii{\p@enumi \theenumi .}\makeatother
\setcounter{enumi}{10}
\item {} 
If you want to cancle your order Click here it will take you to {\hyperref[\detokenize{index:id9}]{\sphinxcrossref{\DUrole{std,std-ref}{Orders:}}}} and follow the guidelines you will come to  know how to cancle your orders.

\item {} 
Click on \sphinxstylestrong{Contact Us} it will show you the \sphinxstylestrong{Tollfree number} and \sphinxstylestrong{Email id}  of company which will be helpful for users.

\item {} 
Click on feedback you will see the below form

\end{enumerate}

\end{description}

\begin{figure}[htbp]
\centering
\capstart

\noindent\sphinxincludegraphics{{feedback}.png}
\caption{User Sending Feedback}\label{\detokenize{index:id21}}\label{\detokenize{index:id11}}\end{figure}

now fill the form and click on \sphinxstylestrong{Send} button thats it.
\begin{enumerate}
\def\theenumi{\arabic{enumi}}
\def\labelenumi{\theenumi .}
\makeatletter\def\p@enumii{\p@enumi \theenumi .}\makeatother
\setcounter{enumi}{13}
\item {} 
Click on FAQ to see for top queries, type your queries in the search field and you will get related response.

\item {} 
TO like this website on \sphinxstylestrong{facebook} Click here and like the facebook page user will be able to see the post.

\item {} 
To download our Mobile app Click here it will direct you to  Google playstore to download our app.

\item {} 
These are the \sphinxstylestrong{Payment Methods} company provides.

\end{enumerate}


\section{Login}
\label{\detokenize{login:login}}\label{\detokenize{login::doc}}
As you entered in the website you have seen some numeric point marked in black over green elipse. For login it is point number 3 click on login then you will see below tab.

\begin{figure}[htbp]
\centering
\capstart

\noindent\sphinxincludegraphics{{adminloginpage}.jpeg}
\caption{Login Page}\label{\detokenize{login:id2}}\label{\detokenize{login:id1}}\end{figure}

Now you are on login page, You can login to this website by choosing one of these three simple way. its all upto you which way you prefer to login.
\begin{quote}
\begin{description}
\item[{These 3 simple ways are :}] \leavevmode\begin{enumerate}
\def\theenumi{\arabic{enumi}}
\def\labelenumi{\theenumi .}
\makeatletter\def\p@enumii{\p@enumi \theenumi .}\makeatother
\item {} 
\sphinxstylestrong{Admin login} which is only for Admin.

\item {} 
By \sphinxstylestrong{google account} for this you have to enter your google \sphinxstyleemphasis{userid} or if you are already loged in to google it will automatically login.

\item {} 
By \sphinxstylestrong{facebook} account for this you just enter your facebook id and password if you are already loged in, it will automatically login or by.

\item {} 
Clicking on \sphinxstylestrong{new user} which will take you to \sphinxstylestrong{Registration Page} so you need register here because you are new user after registration you can login.

\item {} 
One another way is, fill your mobile number in \sphinxstylestrong{mobile number/Username} textfield and then click one \sphinxstyleemphasis{Send OTP} you will get an message including OTP enter that 4 digit OTP(one time password)  in your OTP textfield and click on signin and now you are signed in.

\item {} 
To login to website as \sphinxstylestrong{Admin} just enter \sphinxstylestrong{mobile number/Username} and \sphinxstylestrong{Password} and Click on \sphinxstylestrong{Signin}

\end{enumerate}

\end{description}
\end{quote}

Now you can access the full website.


\subsection{Registration}
\label{\detokenize{register:registration}}\label{\detokenize{register::doc}}
Are you looking for how to register to this website. As you entered in the website you have seen some numeric point, marked in black over green elipse. For registration it is point number 4 click on login then you will see below tab.

\begin{figure}[htbp]
\centering

\noindent\sphinxincludegraphics{{register}.jpeg}
\end{figure}

\begin{figure}[htbp]
\centering
\capstart

\noindent\sphinxincludegraphics{{enterotp}.jpeg}
\caption{Registration and Customer Support Page}\label{\detokenize{register:id3}}\label{\detokenize{register:id1}}\end{figure}

Now you are in registration page, you can register yourself in just few clicks:
\begin{enumerate}
\def\theenumi{\arabic{enumi}}
\def\labelenumi{\theenumi .}
\makeatletter\def\p@enumii{\p@enumi \theenumi .}\makeatother
\item {} 
Enter your \sphinxstylestrong{mobile number}.

\item {} 
Click on \sphinxstylestrong{checkbox}.

\item {} 
Click the \sphinxstylestrong{SUBMIT} button. it will send an \sphinxstyleemphasis{OTP} to your entered mobile number.

\item {} 
Fill \sphinxstylestrong{Mobile OTP}.

\item {} 
Click on \sphinxstylestrong{VERIFY AND LOGIN}.

\end{enumerate}


\subsubsection{Chat with customer care}
\label{\detokenize{register:chat-with-customer-care}}\label{\detokenize{register:id2}}\label{\detokenize{register:my-reference-label}}
Here is one more Amazing feature of this website you can just start the chat with our customer care executive.
\begin{enumerate}
\def\theenumi{\arabic{enumi}}
\def\labelenumi{\theenumi .}
\makeatletter\def\p@enumii{\p@enumi \theenumi .}\makeatother
\setcounter{enumi}{5}
\item {} 
for this you have to go on \sphinxstylestrong{Online}

\end{enumerate}

You will find customer care executive at \sphinxincludegraphics{{werhere}.png}.
\begin{enumerate}
\def\theenumi{\arabic{enumi}}
\def\labelenumi{\theenumi .}
\makeatletter\def\p@enumii{\p@enumi \theenumi .}\makeatother
\setcounter{enumi}{6}
\item {} 
You can also start a \sphinxstylestrong{chat} with customer care executive, They will answer all your question related to online shoping over \sphinxstylestrong{Monomerce}.

\item {} 
You will get several chat option by

\end{enumerate}

clicking on \sphinxincludegraphics{{chatoptions}.png} icon.
\begin{enumerate}
\def\theenumi{\arabic{enumi}}
\def\labelenumi{\theenumi .}
\makeatletter\def\p@enumii{\p@enumi \theenumi .}\makeatother
\setcounter{enumi}{8}
\item {} 
If you want to close the chat tab then you can

\end{enumerate}

clicking on \sphinxincludegraphics{{close}.png} icon.


\paragraph{How to customise Your Chat}
\label{\detokenize{custChatoption:how-to-customise-your-chat}}\label{\detokenize{custChatoption::doc}}
\begin{figure}[htbp]
\centering
\capstart

\noindent\sphinxincludegraphics{{custChatoption}.jpeg}
\caption{Customise your chat pannel}\label{\detokenize{custChatoption:id2}}\label{\detokenize{custChatoption:id1}}\end{figure}

Here you can customise your chat options.
\begin{enumerate}
\def\theenumi{\arabic{enumi}}
\def\labelenumi{\theenumi .}
\makeatletter\def\p@enumii{\p@enumi \theenumi .}\makeatother
\item {} 
To change \sphinxstylestrong{name}.

\item {} 
To \sphinxstylestrong{Emailtranscript}.

\item {} 
You can \sphinxstylestrong{on} or \sphinxstylestrong{off} chat sound.

\item {} 
To \sphinxstylestrong{popout Widget} it will take you to bigger chat field where you can read you entire conversation very comfortably.

\item {} 
You will be able to end your chat session by clicking on \sphinxstylestrong{End This Chat Session}.

\end{enumerate}


\subsection{admin loged in}
\label{\detokenize{adminlogedin:admin-loged-in}}\label{\detokenize{adminlogedin::doc}}
Let’s see what admin can do ?

\begin{figure}[htbp]
\centering
\capstart

\noindent\sphinxincludegraphics{{adminlogdin}.png}
\caption{Admin Portal}\label{\detokenize{adminlogedin:id3}}\label{\detokenize{adminlogedin:id1}}\end{figure}

Now you loged in as \sphinxstylestrong{Admin} so you can configure the several parts of this website, Let’s discuss one by one.
\begin{enumerate}
\def\theenumi{\arabic{enumi}}
\def\labelenumi{\theenumi .}
\makeatletter\def\p@enumii{\p@enumi \theenumi .}\makeatother
\item {} 
this is your \sphinxstylestrong{admin} portal now its time to explore it.

\item {} 
To manage your bussiness in easiest way let’s go through \sphinxstylestrong{Business Management}.

\item {} 
You will get several Business Management options like Ecommerce, Point of sale and product inventory by

\end{enumerate}

clicking on \sphinxincludegraphics{{bmoptions}.png} icon.
\begin{enumerate}
\def\theenumi{\arabic{enumi}}
\def\labelenumi{\theenumi .}
\makeatletter\def\p@enumii{\p@enumi \theenumi .}\makeatother
\setcounter{enumi}{3}
\item {} 
This is \sphinxstylestrong{Blog} section of website, here admin can post some blogs and can manage it too.

\item {} 
This will help admin to \sphinxstylestrong{Manage Users} we will see in details on seprate portal.

\item {} 
Here admin can \sphinxstylestrong{setting} for most part of website. let’s see it on setting portal.

\item {} 
Here admin can \sphinxstylestrong{Save Files} related to their bussiness. we explore it too on Save file portal.

\end{enumerate}


\subsubsection{Save Files}
\label{\detokenize{adminlogedin:save-files}}
\begin{figure}[htbp]
\centering
\capstart

\noindent\sphinxincludegraphics{{savefiles}.png}
\caption{Saving files}\label{\detokenize{adminlogedin:id4}}\label{\detokenize{adminlogedin:id2}}\end{figure}

Here you can save the useful files
\begin{quote}
\begin{enumerate}
\def\theenumi{\alph{enumi}}
\def\labelenumi{\theenumi .}
\makeatletter\def\p@enumii{\p@enumi \theenumi .}\makeatother
\item {} 
Here \sphinxstylestrong{browse} the file from your computer.

\item {} 
select it and Click 0n \sphinxstylestrong{save} button to save it.

\item {} 
Here you can set the type of Files if it is \sphinxstylestrong{static} then turn the toggle button in \sphinxstyleemphasis{static} side if it is \sphinxstylestrong{media} file the turn the toggle button in \sphinxstyleemphasis{media} side.

\end{enumerate}
\begin{enumerate}
\def\theenumi{\arabic{enumi}}
\def\labelenumi{\theenumi .}
\makeatletter\def\p@enumii{\p@enumi \theenumi .}\makeatother
\setcounter{enumi}{7}
\item {} 
Here admin can right some queries \sphinxcode{\sphinxupquote{@aks}} to search their user like \sphinxstylestrong{akshay sinha} and can chat with searched user.

\item {} 
Here admin can read \sphinxstylestrong{Messages} and send too.

\item {} 
Here admin will see all the \sphinxstylestrong{notifications}.

\item {} 
Here admin will come to know about some more options for admin profile like \sphinxstylestrong{Setting, About and Logout}.

\end{enumerate}
\end{quote}


\paragraph{Business Management}
\label{\detokenize{businessmgmt:business-management}}\label{\detokenize{businessmgmt::doc}}
\begin{figure}[htbp]
\centering

\noindent\sphinxincludegraphics{{BMmon}.jpeg}
\end{figure}

\begin{figure}[htbp]
\centering
\capstart

\noindent\sphinxincludegraphics{{BMmon1}.jpeg}
\caption{Your Business Report}\label{\detokenize{businessmgmt:id2}}\label{\detokenize{businessmgmt:id1}}\end{figure}

Here now you are on \sphinxstylestrong{Business Management} portal. you can see the graph of your \sphinxstylestrong{Online} business as well as \sphinxstylestrong{Offline} bussiness.
Here you have Three options like:
\begin{enumerate}
\def\theenumi{\arabic{enumi}}
\def\labelenumi{\theenumi .}
\makeatletter\def\p@enumii{\p@enumi \theenumi .}\makeatother
\item {} 
Here you can see the \sphinxcode{\sphinxupquote{Graph}}, \sphinxcode{\sphinxupquote{Pai Chart}}, \sphinxstyleemphasis{new order}, \sphinxstyleemphasis{new customer}, \sphinxstyleemphasis{total collections} and \sphinxstyleemphasis{total sale} of your business \sphinxstylestrong{Monthly} as well as.

\item {} 
\sphinxstylestrong{Weekly}.

\item {} 
and on \sphinxstylestrong{Daily} basis.

\end{enumerate}

\begin{figure}[htbp]
\centering

\noindent\sphinxincludegraphics{{BMweek}.jpeg}
\end{figure}

In the above image there is \sphinxstylestrong{Weekly} business data.


\subparagraph{Business Management Ecommerce}
\label{\detokenize{BMecomm:business-management-ecommerce}}\label{\detokenize{BMecomm::doc}}
\begin{figure}[htbp]
\centering

\noindent\sphinxincludegraphics{{BMecomm1}.jpeg}
\end{figure}

Now you are on the subportal of \sphinxstyleemphasis{Business management} which is \sphinxstylestrong{Ecommerce}. here also we have created an easiest tool to analyize your Business. which will give you the analyized data of you business
\begin{enumerate}
\def\theenumi{\arabic{enumi}}
\def\labelenumi{\theenumi .}
\makeatletter\def\p@enumii{\p@enumi \theenumi .}\makeatother
\item {} 
for \sphinxstylestrong{Monthly}

\item {} 
for \sphinxstylestrong{Weekly} and

\item {} 
for \sphinxstylestrong{Daily} as well.

\end{enumerate}

\begin{figure}[htbp]
\centering

\noindent\sphinxincludegraphics{{BMecomm2}.jpeg}
\end{figure}

The above image is an example of \sphinxstylestrong{weekly} analyized data. In the same way you can check for \sphinxstylestrong{Day} too.

Now let’s see the major settings of \sphinxstyleemphasis{Ecommerce} portal:


\subparagraph{Listing}
\label{\detokenize{listing:listing}}\label{\detokenize{listing::doc}}
\begin{figure}[htbp]
\centering
\capstart

\noindent\sphinxincludegraphics{{listings}.jpeg}
\caption{Listing Products}\label{\detokenize{listing:id7}}\label{\detokenize{listing:id1}}\end{figure}

By clicking on \sphinxstylestrong{Listing} you get the above portal. Here you can
\begin{enumerate}
\def\theenumi{\arabic{enumi}}
\def\labelenumi{\theenumi .}
\makeatletter\def\p@enumii{\p@enumi \theenumi .}\makeatother
\item {} 
\sphinxstylestrong{Delete} the product and

\item {} 
\sphinxstylestrong{Edit} the product.

\end{enumerate}

\begin{figure}[htbp]
\centering
\capstart

\noindent\sphinxincludegraphics{{editlisting}.jpeg}
\caption{Editing And Creating Products}\label{\detokenize{listing:id8}}\label{\detokenize{listing:id2}}\end{figure}

Here you fill above field to edit.
\begin{enumerate}
\def\theenumi{\arabic{enumi}}
\def\labelenumi{\theenumi .}
\makeatletter\def\p@enumii{\p@enumi \theenumi .}\makeatother
\item {} 
Fill \sphinxstylestrong{Category}.

\item {} 
Edit \sphinxstylestrong{Product}.

\item {} 
Set the product \sphinxstylestrong{Index}.

\item {} 
Add a media file as per your convenienc e.i \sphinxstyleemphasis{image, document, video or link}.

\item {} 
Here you can do the \sphinxstylestrong{Indexing} of your product’s image to show different different phase of product.

\item {} 
Here you can write the \sphinxstylestrong{Description} for your particular product.

\end{enumerate}


\subparagraph{Edit Product}
\label{\detokenize{listing:edit-product}}
\begin{figure}[htbp]
\centering
\capstart

\noindent\sphinxincludegraphics{{editprod}.jpeg}
\caption{Editing Products}\label{\detokenize{listing:id9}}\label{\detokenize{listing:id3}}\end{figure}
\begin{description}
\item[{Here you can edit product details like price, description, discount, quantity etc. How ? let’s see.}] \leavevmode\begin{enumerate}
\def\theenumi{\arabic{enumi}}
\def\labelenumi{\theenumi .}
\makeatletter\def\p@enumii{\p@enumi \theenumi .}\makeatother
\item {} 
Here you can edit the \sphinxstylestrong{Product Name}.

\item {} 
Here you set the \sphinxstylestrong{MRP} for this product either by clicking on UP and DOWN arrow symbol or by typing in this textfield.

\item {} 
Here you set the product \sphinxstylestrong{Logistics} in same way as for MRP.

\item {} 
Here you set the product \sphinxstylestrong{Cost} in same way as for MRP.

\item {} 
Here you set the \sphinxstylestrong{Discount Offers} in percentage.

\item {} 
Here you can set the \sphinxstylestrong{Display Picture} of product by browsing.

\item {} 
Here you set \sphinxstylestrong{SKU} (stock keeping unit).

\item {} 
Here you set \sphinxstylestrong{Serial id} of product.

\item {} 
Here you set \sphinxstylestrong{Reorder Threshold} (is the minimum count of an item you keep on hand).

\item {} 
In this textarea you fill \sphinxstylestrong{Product Meta}.

\item {} 
Here you can specify the \sphinxstylestrong{Gross Weight} (Maximum Quantity limit to order for customer).

\item {} 
Here you set the \sphinxstylestrong{Unit} of product e.i. \sphinxstyleemphasis{gram, kilogram, liter, milileter} etc.

\item {} 
Here you can set the \sphinxstylestrong{How Much} (in unit) for example 500gm, 250gm etc.

\item {} 
To \sphinxstylestrong{Save} your edited values you havee click on save button.

\item {} 
Here you can set \sphinxstylestrong{Alias} (assumed identity) of your product.

\item {} 
In thid textfield you write the \sphinxstylestrong{Description} for product.

\item {} 
Here you have one more option to set \sphinxstylestrong{Secondary SKU} (stock keeping unit) e.i. \sphinxstyleemphasis{product varient} it will help in product categorization.

\item {} 
Here you set the \sphinxstylestrong{Serial Id} for product’s category.

\item {} 
Now set \sphinxstylestrong{price} in Rupees and paise.

\item {} 
Now click on \sphinxstylestrong{Add} button to add your product in category.

\item {} 
Here you can \sphinxstylestrong{seach and select} product then

\item {} 
Set the quantity by typing or clicking on given arrows and

\end{enumerate}

\end{description}

then
\begin{enumerate}
\def\theenumi{\arabic{enumi}}
\def\labelenumi{\theenumi .}
\makeatletter\def\p@enumii{\p@enumi \theenumi .}\makeatother
\setcounter{enumi}{22}
\item {} 
Clicking on \sphinxincludegraphics{{add}.png} button to add you product in list.

\end{enumerate}

\begin{figure}[htbp]
\centering
\capstart

\noindent\sphinxincludegraphics{{createnew}.jpeg}
\caption{Creating List}\label{\detokenize{listing:id10}}\label{\detokenize{listing:id4}}\end{figure}

In Ecommerce’s \sphinxstyleemphasis{Listing} Portal
\begin{enumerate}
\def\theenumi{\arabic{enumi}}
\def\labelenumi{\theenumi .}
\makeatletter\def\p@enumii{\p@enumi \theenumi .}\makeatother
\item {} 
Click on \sphinxstylestrong{New} to create a new list.

\item {} 
If you don’t want to create list, just click on \sphinxstylestrong{Go back} button to return on previous page. if you want to continue then

\item {} 
Fill this textarea with \sphinxstylestrong{Product Name} then

\item {} 
Set the product \sphinxstylestrong{Index} .

\item {} 
\sphinxstylestrong{Add a Media file} for product.

\item {} 
This is optional if you want add some \sphinxstylestrong{Description} for product then fill this textarea else leave as it is.

\item {} 
At last click on \sphinxstylestrong{Submit} button and your product will be add in list.

\end{enumerate}

\begin{figure}[htbp]
\centering
\capstart

\noindent\sphinxincludegraphics{{browse}.jpeg}
\caption{Products list}\label{\detokenize{listing:id11}}\label{\detokenize{listing:id5}}\end{figure}

Here:
\begin{enumerate}
\def\theenumi{\arabic{enumi}}
\def\labelenumi{\theenumi .}
\makeatletter\def\p@enumii{\p@enumi \theenumi .}\makeatother
\item {} 
you can \sphinxstylestrong{Browse} your products.

\item {} 
Here you can see the products in  \sphinxstylestrong{Grid} view.

\item {} 
By clicking here you will get the \sphinxstylestrong{List} view of product same as you are seeing above.

\end{enumerate}

by    4. clicking on \sphinxincludegraphics{{pto}.png} you can chenge the page.
\begin{enumerate}
\def\theenumi{\arabic{enumi}}
\def\labelenumi{\theenumi .}
\makeatletter\def\p@enumii{\p@enumi \theenumi .}\makeatother
\setcounter{enumi}{4}
\item {} 
Here you can select items.

\item {} 
This gives you \sphinxstylestrong{Delete} option.

\item {} 
This gives you \sphinxstylestrong{Edit} option Same as we had done on {\hyperref[\detokenize{listing:edit-product}]{\sphinxcrossref{Edit Product}}} .

\item {} 
Here you can upload your product list in \sphinxstylestrong{Bulk} just by selecting the Excel file.

\end{enumerate}

\begin{figure}[htbp]
\centering
\capstart

\noindent\sphinxincludegraphics{{bulklistcreation}.jpeg}
\caption{Bulk List creation}\label{\detokenize{listing:id12}}\label{\detokenize{listing:id6}}\end{figure}

Here you:
\begin{enumerate}
\def\theenumi{\alph{enumi}}
\def\labelenumi{\theenumi .}
\makeatletter\def\p@enumii{\p@enumi \theenumi .}\makeatother
\item {} 
\sphinxstylestrong{Browse} your Excel file and

\item {} 
Click on \sphinxstylestrong{Upload} it will save your bulky data in lists just in few clicks.

\end{enumerate}


\subparagraph{Configure}
\label{\detokenize{configure:configure}}\label{\detokenize{configure::doc}}
\begin{figure}[htbp]
\centering
\capstart

\noindent\sphinxincludegraphics{{configure}.jpeg}
\caption{Configuring your website}\label{\detokenize{configure:id9}}\label{\detokenize{configure:id1}}\end{figure}

Let’s see \sphinxstylestrong{Configure} portal.
\begin{enumerate}
\def\theenumi{\arabic{enumi}}
\def\labelenumi{\theenumi .}
\makeatletter\def\p@enumii{\p@enumi \theenumi .}\makeatother
\item {} 
Here you will configure the \sphinxstyleemphasis{Fields or Products} by clicking on \sphinxstylestrong{New}.

\end{enumerate}

Here
\begin{quote}
\begin{enumerate}
\def\theenumi{\alph{enumi}}
\def\labelenumi{\theenumi .}
\makeatletter\def\p@enumii{\p@enumi \theenumi .}\makeatother
\item {} 
This is a dropdown portion where you will get 2 options \sphinxstylestrong{Filed and Products} , choose which one you want to configure for e.g let’s select \sphinxcode{\sphinxupquote{fields}}.

\item {} 
This one is also a dropdown portion here you will get options to set the \sphinxstylestrong{Type} like \sphinxstyleemphasis{char, float, boolean, date} etc.

\item {} 
Here you will set the \sphinxstylestrong{name of the field} for e.g. \sphinxstyleemphasis{length} etc.

\item {} 
Here you have to set the \sphinxstylestrong{Unit} of the field for e.g. \sphinxstyleemphasis{mm} (milimeter).

\item {} 
Here you can write some helpful \sphinxstylestrong{Text} which will give some information about the field.

\item {} 
Here you will set some \sphinxstylestrong{Default value} for specific field for e.g. \sphinxstyleemphasis{mm} (milimeter).

\item {} 
Now you are done just click on \sphinxstylestrong{Submit} and your field is set.

\end{enumerate}
\begin{enumerate}
\def\theenumi{\arabic{enumi}}
\def\labelenumi{\theenumi .}
\makeatletter\def\p@enumii{\p@enumi \theenumi .}\makeatother
\setcounter{enumi}{1}
\item {} 
Here you can see your configured \sphinxstylestrong{Fields} .

\item {} 
Here you will be able to \sphinxstyleemphasis{see and configure} the \sphinxstylestrong{Products} .

\end{enumerate}
\end{quote}


\subparagraph{Products}
\label{\detokenize{configure:products}}
\begin{figure}[htbp]
\centering
\capstart

\noindent\sphinxincludegraphics{{prods}.jpeg}
\caption{Managing Products}\label{\detokenize{configure:id10}}\label{\detokenize{configure:id2}}\end{figure}

Here you will get following options:
\begin{enumerate}
\def\theenumi{\arabic{enumi}}
\def\labelenumi{\theenumi .}
\makeatletter\def\p@enumii{\p@enumi \theenumi .}\makeatother
\item {} 
To \sphinxstylestrong{Edit} genric product just click here and you will see:

\end{enumerate}

\begin{figure}[htbp]
\centering
\capstart

\noindent\sphinxincludegraphics{{editgenricprod}.jpeg}
\caption{Edit Genric Products}\label{\detokenize{configure:id11}}\label{\detokenize{configure:id3}}\end{figure}

Here you have to follow the following steps:
\begin{quote}
\begin{enumerate}
\def\theenumi{\alph{enumi}}
\def\labelenumi{\theenumi .}
\makeatletter\def\p@enumii{\p@enumi \theenumi .}\makeatother
\item {} 
Textarea \sphinxstylestrong{Parent} is to fill genric prduct name.

\item {} 
Fill the product name for e.g \sphinxstylestrong{Grain}.

\item {} 
Here you can specify the \sphinxstylestrong{Property Fields}.

\item {} 
Here you can set the \sphinxstylestrong{Min Cost} of product.

\item {} 
If you will enable \sphinxstylestrong{Restricted} for fixed minimum cost it will not direct for sale.

\item {} 
Here you can set image which will be \sphinxstylestrong{Visual} for product.

\item {} 
Here you can set \sphinxstylestrong{Banner} image.

\item {} 
To save these all editing click on \sphinxstylestrong{save} .

\end{enumerate}
\begin{enumerate}
\def\theenumi{\arabic{enumi}}
\def\labelenumi{\theenumi .}
\makeatletter\def\p@enumii{\p@enumi \theenumi .}\makeatother
\setcounter{enumi}{1}
\item {} 
if you want to \sphinxstylestrong{Delete} then click here.

\item {} 
To \sphinxstylestrong{Explore} product click here and you will see:

\end{enumerate}
\end{quote}

\begin{figure}[htbp]
\centering

\noindent\sphinxincludegraphics{{prodexp}.jpeg}
\end{figure}

Here you will get seprate tab for each product as you can see in above image \sphinxstylestrong{a}, \sphinxstylestrong{b} and \sphinxstylestrong{c} are pointing those \sphinxstylestrong{Tab}. You can switch between these tabs just by clicking on it.
\begin{enumerate}
\def\theenumi{\arabic{enumi}}
\def\labelenumi{\theenumi .}
\makeatletter\def\p@enumii{\p@enumi \theenumi .}\makeatother
\setcounter{enumi}{3}
\item {} 
It will take you to \sphinxstylestrong{Offer Banners} where you can configure the offers.

\end{enumerate}

\begin{figure}[htbp]
\centering
\capstart

\noindent\sphinxincludegraphics{{crtofferbanner}.jpeg}
\caption{Creating Offer Banner}\label{\detokenize{configure:id12}}\label{\detokenize{configure:id4}}\end{figure}

Here
\begin{quote}
\begin{enumerate}
\def\theenumi{\arabic{enumi}}
\def\labelenumi{\theenumi .}
\makeatletter\def\p@enumii{\p@enumi \theenumi .}\makeatother
\item {} 
you fill \sphinxstylestrong{Title} of \sphinxstyleemphasis{banner}.

\item {} 
Here you fill \sphinxstylestrong{Subtitle} of \sphinxstyleemphasis{banner}.

\item {} 
Here you set the \sphinxstylestrong{Banner} Image (in landscape mode).

\item {} 
Here you set the \sphinxstylestrong{Potrait Banner} Image (in landscape mode).

\item {} 
Now you can customise the \sphinxstylestrong{Level} of images(banners) for e.g Enter 1 if it will be on the slide show , 2 if it will be shown in the sidebar, 3 if this will be a flash message.

\item {} 
Now its time to set the \sphinxstylestrong{Page} where you want to show this banner.

\item {} 
Now click on \sphinxstylestrong{Create} and your banner is ready to visual on your website.

\item {} 
Here you can \sphinxstylestrong{Search} your created banner.

\item {} 
In this section your created banner will be appear, where you can \sphinxstyleemphasis{edit, delete and explore} it.

\end{enumerate}
\begin{enumerate}
\def\theenumi{\arabic{enumi}}
\def\labelenumi{\theenumi .}
\makeatletter\def\p@enumii{\p@enumi \theenumi .}\makeatother
\setcounter{enumi}{4}
\item {} 
By clicking here you will be on \sphinxstylestrong{Promocodes} browser here you can configure promocodes.

\end{enumerate}
\end{quote}

\begin{figure}[htbp]
\centering
\capstart

\noindent\sphinxincludegraphics{{crtpromocode}.jpeg}
\caption{Creating Promocode}\label{\detokenize{configure:id13}}\label{\detokenize{configure:id5}}\end{figure}

Here
\begin{quote}
\begin{enumerate}
\def\theenumi{\arabic{enumi}}
\def\labelenumi{\theenumi .}
\makeatletter\def\p@enumii{\p@enumi \theenumi .}\makeatother
\item {} 
Set the \sphinxstylestrong{Name} of promocode for e.g NEW50

\item {} 
Set \sphinxstylestrong{Discount \%} for e.g 5\%.

\item {} 
Limit \sphinxstylestrong{Number Of Times} for this promocode for e.g \sphinxcode{\sphinxupquote{1}} bcz it is only for new user.

\item {} 
Here you have to set \sphinxstylestrong{Valid till} date and time.

\item {} 
Now click on \sphinxstylestrong{Save} butoon to create your promocode.

\item {} 
Here you can \sphinxstylestrong{Search} your created promocode by their name.

\item {} 
In this section your created promocode will be appear where you can \sphinxstyleemphasis{edit delete and explore} your promocode.

\end{enumerate}
\begin{enumerate}
\def\theenumi{\arabic{enumi}}
\def\labelenumi{\theenumi .}
\makeatletter\def\p@enumii{\p@enumi \theenumi .}\makeatother
\setcounter{enumi}{5}
\item {} 
By clicking here you will be on \sphinxstylestrong{FAQ Questions} browser here you can see  ask and answer the FAQs.

\end{enumerate}
\end{quote}

\begin{figure}[htbp]
\centering
\capstart

\noindent\sphinxincludegraphics{{FAQ}.jpeg}
\caption{Frequently Asked questions and answers about Your Website.}\label{\detokenize{configure:id14}}\label{\detokenize{configure:id6}}\end{figure}

Here:
\begin{quote}
\begin{enumerate}
\def\theenumi{\arabic{enumi}}
\def\labelenumi{\theenumi .}
\makeatletter\def\p@enumii{\p@enumi \theenumi .}\makeatother
\item {} 
You can write \sphinxstylestrong{Question} which may come in user’s mind.

\item {} 
And you provide them \sphinxstylestrong{Answer} too, To solve some queries and make it easier for your customes.

\item {} 
Click on \sphinxstylestrong{Save} it will be appear in FAQ section.

\item {} 
Here you can \sphinxstylestrong{Search} the posted questions \sphinxstyleemphasis{by question} to check their answer.

\end{enumerate}
\begin{enumerate}
\def\theenumi{\arabic{enumi}}
\def\labelenumi{\theenumi .}
\makeatletter\def\p@enumii{\p@enumi \theenumi .}\makeatother
\setcounter{enumi}{6}
\item {} 
Here you can check your \sphinxstylestrong{Service Area} and Add too.

\end{enumerate}
\end{quote}

\begin{figure}[htbp]
\centering
\capstart

\noindent\sphinxincludegraphics{{servicearea}.jpeg}
\caption{Configuring Service Area}\label{\detokenize{configure:id15}}\label{\detokenize{configure:id7}}\end{figure}

Here you can specify your service area:
\begin{quote}
\begin{enumerate}
\def\theenumi{\arabic{enumi}}
\def\labelenumi{\theenumi .}
\makeatletter\def\p@enumii{\p@enumi \theenumi .}\makeatother
\item {} 
In this textarea fill the pin code of the area which you want to add and click on \sphinxstylestrong{Add} it will be added in your service area.

\item {} 
Here you can check your \sphinxstylestrong{Service area} and delete too.

\end{enumerate}
\begin{enumerate}
\def\theenumi{\arabic{enumi}}
\def\labelenumi{\theenumi .}
\makeatletter\def\p@enumii{\p@enumi \theenumi .}\makeatother
\setcounter{enumi}{7}
\item {} 
This is your websites \sphinxstylestrong{Images} browser here you can set the image for different different portions of website.

\end{enumerate}
\end{quote}

\begin{figure}[htbp]
\centering
\capstart

\noindent\sphinxincludegraphics{{image}.jpeg}
\caption{Setting Images for your website}\label{\detokenize{configure:id16}}\label{\detokenize{configure:id8}}\end{figure}

Here you can set images for your website’s different different portions:
\begin{enumerate}
\def\theenumi{\arabic{enumi}}
\def\labelenumi{\theenumi .}
\makeatletter\def\p@enumii{\p@enumi \theenumi .}\makeatother
\item {} 
Here set \sphinxstylestrong{Background Image} (cover Image For Login Page).

\item {} 
Here set \sphinxstylestrong{Cart Image} (Right Side Displaying Cart Image).

\item {} 
Here set \sphinxstylestrong{Payment Image} which will be appear in the footer section of website.

\item {} 
Here set \sphinxstylestrong{Payment Potrait Image} for mobile site.

\item {} 
Here set \sphinxstylestrong{Search Background Image} which will be appear in Header Search Background.

\item {} 
Here you can set \sphinxstylestrong{Blog Background Image} for Blog Page.

\item {} 
Click on \sphinxstylestrong{Save} button and images will be saved.

\end{enumerate}


\subparagraph{Orders}
\label{\detokenize{orders:orders}}\label{\detokenize{orders::doc}}
\begin{figure}[htbp]
\centering
\capstart

\noindent\sphinxincludegraphics{{orders}.png}
\caption{Orders}\label{\detokenize{orders:id4}}\label{\detokenize{orders:id1}}\end{figure}

Let’s see \sphinxstylestrong{How to Manage Orders} portal.
\begin{enumerate}
\def\theenumi{\arabic{enumi}}
\def\labelenumi{\theenumi .}
\makeatletter\def\p@enumii{\p@enumi \theenumi .}\makeatother
\item {} 
Here you can change the pages to browse \sphinxstylestrong{Order} portal.

\item {} 
Here is order pannel by selecting \sphinxstyleemphasis{order id} you can check the \sphinxstylestrong{order details} .

\end{enumerate}


\subparagraph{Approve Order}
\label{\detokenize{orders:approve-order}}
\begin{figure}[htbp]
\centering
\capstart

\noindent\sphinxincludegraphics{{orderapprove}.png}
\caption{Approving Orders}\label{\detokenize{orders:id5}}\label{\detokenize{orders:id2}}\end{figure}

Above is order details view. In the \sphinxstyleemphasis{green eliptical circle} you can see that \sphinxstyleemphasis{Amount : rs50} but \sphinxstyleemphasis{Paid Amount : rs0} means transaction is failed, rarely it happens in online transaction. so if the customer will complain about it after verifing the payment in your account you can approve the payment (order) by clicking on \sphinxstylestrong{Approve} and order will be placed.
\begin{enumerate}
\def\theenumi{\arabic{enumi}}
\def\labelenumi{\theenumi .}
\makeatletter\def\p@enumii{\p@enumi \theenumi .}\makeatother
\item {} 
Here you can see the \sphinxstylestrong{Order Details} view.

\item {} 
By clicking on \sphinxstyleemphasis{approve} you can \sphinxstylestrong{Approve} the order.

\item {} 
By clicking on \sphinxstyleemphasis{reject} you can \sphinxstylestrong{Reject} .

\item {} 
Here you can write \sphinxstylestrong{Logs}

\end{enumerate}

so click on \sphinxincludegraphics{{logs}.png}.
\begin{enumerate}
\def\theenumi{\arabic{enumi}}
\def\labelenumi{\theenumi .}
\makeatletter\def\p@enumii{\p@enumi \theenumi .}\makeatother
\setcounter{enumi}{4}
\item {} 
Click on \sphinxstylestrong{Change status} and give the conformation to change the status of order,  it will change like \sphinxstyleemphasis{Packed to shipped, shipped to in transit, intransit to reached nearest hub} etc.

\item {} 
Click on \sphinxstylestrong{Manifest} button to generate manifest.

\end{enumerate}

\begin{figure}[htbp]
\centering
\capstart

\noindent\sphinxincludegraphics{{gnrtmanifest}.png}
\caption{Generating menifest}\label{\detokenize{orders:id6}}\label{\detokenize{orders:id3}}\end{figure}

Above tab is generate manifest tab.
\begin{quote}
\begin{quote}
\begin{enumerate}
\def\theenumi{\arabic{enumi}}
\def\labelenumi{\theenumi .}
\makeatletter\def\p@enumii{\p@enumi \theenumi .}\makeatother
\item {} 
In this textarea fill the \sphinxstylestrong{Courier Name} for e.g \sphinxstyleemphasis{Ekart} .

\item {} 
In this textarea fill \sphinxstylestrong{Courier AWB(airwaybill) Number} .

\item {} 
Here you can write \sphinxstylestrong{Description} .

\item {} 
And click on \sphinxstylestrong{Save} button and manifest is ready.

\end{enumerate}
\begin{enumerate}
\def\theenumi{\arabic{enumi}}
\def\labelenumi{\theenumi .}
\makeatletter\def\p@enumii{\p@enumi \theenumi .}\makeatother
\setcounter{enumi}{6}
\item {} 
Click on \sphinxstylestrong{Delete} button to \sphinxstyleemphasis{delete} the generated manifest.

\item {} 
And click on \sphinxstylestrong{Submit} button.

\end{enumerate}
\end{quote}
\begin{enumerate}
\def\theenumi{\arabic{enumi}}
\def\labelenumi{\theenumi .}
\makeatletter\def\p@enumii{\p@enumi \theenumi .}\makeatother
\setcounter{enumi}{2}
\item {} 
This one is \sphinxstylestrong{Search Field} here you can search order \sphinxstylestrong{by status} for e.g \sphinxstyleemphasis{packed, cancelled} .

\end{enumerate}
\end{quote}


\subparagraph{Support}
\label{\detokenize{support:support}}\label{\detokenize{support:id1}}\label{\detokenize{support::doc}}
\begin{figure}[htbp]
\centering

\noindent\sphinxincludegraphics{{support}.jpeg}
\end{figure}

Now you are here to do the setting of support.
\begin{enumerate}
\def\theenumi{\arabic{enumi}}
\def\labelenumi{\theenumi .}
\makeatletter\def\p@enumii{\p@enumi \theenumi .}\makeatother
\item {} 
Here you can see the numbers of request on \sphinxstylestrong{Request Tab}.

\item {} 
This is \sphinxstylestrong{Search} field for support system where you can check the list of requests but you have to \sphinxstyleemphasis{search by status} for e.g if you will type \sphinxcode{\sphinxupquote{solved}} then all the solved request will be appear here.

\item {} 
This is \sphinxstylestrong{Delete} button to delete the request.

\item {} 
This will direct you to \sphinxstylestrong{Support Details} .

\end{enumerate}

\begin{figure}[htbp]
\centering

\noindent\sphinxincludegraphics{{requestdetails}.jpeg}
\end{figure}

Above image and description will help you to solve the request and to change the request status. How ! let’s see:
\begin{quote}
\begin{enumerate}
\def\theenumi{\arabic{enumi}}
\def\labelenumi{\theenumi .}
\makeatletter\def\p@enumii{\p@enumi \theenumi .}\makeatother
\item {} 
Here you are seeing the \sphinxstylestrong{Request Details} like \sphinxstyleemphasis{email, mobile and messages} .

\item {} 
Here you will \sphinxstylestrong{Response} to that request.

\item {} 
And click on \sphinxstylestrong{Send} to send the response.

\item {} 
Here you can change the \sphinxstylestrong{Status} of the request for e.g \sphinxstyleemphasis{created, ongoing, resolved and junk} .

\end{enumerate}
\begin{enumerate}
\def\theenumi{\arabic{enumi}}
\def\labelenumi{\theenumi .}
\makeatletter\def\p@enumii{\p@enumi \theenumi .}\makeatother
\setcounter{enumi}{4}
\item {} 
This is a button to \sphinxstylestrong{Switch} the pages.

\item {} 
This is a button to \sphinxstylestrong{Refresh} the pages.

\end{enumerate}
\end{quote}


\subparagraph{Pages}
\label{\detokenize{pages:pages}}\label{\detokenize{pages::doc}}
\begin{figure}[htbp]
\centering
\capstart

\noindent\sphinxincludegraphics{{pages}.jpeg}
\caption{Creating several types of pages for your business.}\label{\detokenize{pages:id2}}\label{\detokenize{pages:id1}}\end{figure}

Now you are on Ecommerce’s pages portal to configure pages for your website. so let’s follow the steps:
\begin{enumerate}
\def\theenumi{\arabic{enumi}}
\def\labelenumi{\theenumi .}
\makeatletter\def\p@enumii{\p@enumi \theenumi .}\makeatother
\item {} 
Click on \sphinxstylestrong{New} to create a new page.

\end{enumerate}

Are you wandering how to create page ? Let’s see:
\begin{quote}
\begin{enumerate}
\def\theenumi{\alph{enumi}}
\def\labelenumi{\theenumi .}
\makeatletter\def\p@enumii{\p@enumi \theenumi .}\makeatother
\item {} 
Give the \sphinxstylestrong{Title} of a new page and

\item {} 
Click on \sphinxstylestrong{Generate} button to generate page’s URL.

\item {} 
Click on the checkbox to appear this page in \sphinxstylestrong{Top Level Menu} .

\item {} 
Click on the checkbox to appear this page in \sphinxstylestrong{Bottom Level Menu} you can enable both too.

\item {} 
Here you have 2 option either you \sphinxstylestrong{Save} the page or if you have done any mistake while creating a page then \sphinxstylestrong{Reset} it and create again. To save the page click on \sphinxstyleemphasis{save} button and to reset click on \sphinxstyleemphasis{reset} button. yes ! this is that simple.

\end{enumerate}
\begin{enumerate}
\def\theenumi{\arabic{enumi}}
\def\labelenumi{\theenumi .}
\makeatletter\def\p@enumii{\p@enumi \theenumi .}\makeatother
\setcounter{enumi}{1}
\item {} 
Click on \sphinxstylestrong{Browse} to browse the created pages.

\end{enumerate}
\end{quote}

\begin{figure}[htbp]
\centering

\noindent\sphinxincludegraphics{{browsepage}.jpeg}
\end{figure}

Above you are seeing the created pages.
\begin{enumerate}
\def\theenumi{\arabic{enumi}}
\def\labelenumi{\theenumi .}
\makeatletter\def\p@enumii{\p@enumi \theenumi .}\makeatother
\item {} 
Here you are seeing the page \sphinxstylestrong{Title} for e.g \sphinxstyleemphasis{Summer offer, About Us, Terms of Use, Privacy Policy} etc.

\item {} 
Here you can \sphinxstylestrong{Search} your created page \sphinxstyleemphasis{by title} (Page Title) for e.g if you have to check  .

\item {} 
Here you have option to \sphinxstylestrong{Edit} the created page its process is same as for \sphinxstyleemphasis{create page} .

\item {} 
Here you can see a \sphinxstylestrong{Delete} button so you can \sphinxstyleemphasis{delete the page} if you do not want it in your website.

\item {} 
By clicking on this button you can see the \sphinxstylestrong{Page Details} as you are seeing below.

\end{enumerate}

\begin{figure}[htbp]
\centering

\noindent\sphinxincludegraphics{{browsepagedetails}.jpeg}
\end{figure}

Once you checked the page click on \sphinxincludegraphics{{close}.png} to close the page.


\subparagraph{Business Management Point Of Sale}
\label{\detokenize{BMpos:business-management-point-of-sale}}\label{\detokenize{BMpos::doc}}
\begin{figure}[htbp]
\centering
\capstart

\noindent\sphinxincludegraphics{{BMpos}.png}
\caption{POS (Point of sale)}\label{\detokenize{BMpos:id4}}\label{\detokenize{BMpos:id1}}\end{figure}

You clicked on pos so you are here now,
\begin{quote}
\begin{enumerate}
\def\theenumi{\arabic{enumi}}
\def\labelenumi{\theenumi .}
\makeatletter\def\p@enumii{\p@enumi \theenumi .}\makeatother
\item {} 
It has taken you to \sphinxstylestrong{POS} (point of sale) portal which is subportal of \sphinxstyleemphasis{Business Management}. Now its time to know POS in detail.

\end{enumerate}
\begin{enumerate}
\def\theenumi{\alph{enumi}}
\def\labelenumi{\theenumi .}
\makeatletter\def\p@enumii{\p@enumi \theenumi .}\makeatother
\item {} 
Here you can check you \sphinxstyleemphasis{point of sale} as per \sphinxstylestrong{Month} . You can check prticular months pos just select the month data will display on same tab.

\item {} 
Here is same feature for \sphinxstylestrong{Week} too.

\item {} 
In the same way you can check \sphinxstyleemphasis{pos} for \sphinxstylestrong{Day}.

\item {} 
Here admin can search \sphinxstylestrong{Customer} by their \sphinxstyleemphasis{Name}.

\end{enumerate}
\end{quote}

\begin{figure}[htbp]
\centering
\capstart

\noindent\sphinxincludegraphics{{BMposcust}.jpeg}
\caption{Customers}\label{\detokenize{BMpos:id5}}\label{\detokenize{BMpos:id2}}\end{figure}

Above image is example of search Customer by Name.
\begin{enumerate}
\def\theenumi{\alph{enumi}}
\def\labelenumi{\theenumi .}
\makeatletter\def\p@enumii{\p@enumi \theenumi .}\makeatother
\setcounter{enumi}{4}
\item {} 
Here admin can search \sphinxstylestrong{Invoice} by their \sphinxstyleemphasis{Id}.

\end{enumerate}

\begin{figure}[htbp]
\centering
\capstart

\noindent\sphinxincludegraphics{{BMposinvoice}.jpeg}
\caption{Invoices}\label{\detokenize{BMpos:id6}}\label{\detokenize{BMpos:id3}}\end{figure}

Above image is example of search Invoice by Id.
\begin{enumerate}
\def\theenumi{\arabic{enumi}}
\def\labelenumi{\theenumi .}
\makeatletter\def\p@enumii{\p@enumi \theenumi .}\makeatother
\setcounter{enumi}{1}
\item {} 
Clicking on it will show you the Product details.

\item {} 
Click here to edit the product it will show you details of that product which will reflect in \sphinxstyleemphasis{Manufacture} section. Here you can change the product details and quantity as per your requirements. It is similar to {\hyperref[\detokenize{listing:edit-product}]{\sphinxcrossref{\DUrole{std,std-ref}{Edit Product}}}} so just try once you will come to know how it works.

\end{enumerate}


\subparagraph{Products Inventory}
\label{\detokenize{prodinventory:products-inventory}}\label{\detokenize{prodinventory::doc}}
\begin{figure}[htbp]
\centering
\capstart

\noindent\sphinxincludegraphics{{prodinventory}.png}
\caption{Product inventory}\label{\detokenize{prodinventory:id9}}\label{\detokenize{prodinventory:id1}}\end{figure}

As you loged in as admin you will see the \sphinxstylestrong{Business Management} click on it then click on \sphinxstylestrong{Products Inventory} you will see the above tab
where you can
\begin{enumerate}
\def\theenumi{\arabic{enumi}}
\def\labelenumi{\theenumi .}
\makeatletter\def\p@enumii{\p@enumi \theenumi .}\makeatother
\item {} 
Click on any product it will be expand as you can see in the above image.

\item {} 
Here is Refresh, Next, Prev buttons for pages bacause there is lots of products, You can use it as per your convenience.

\item {} 
Click here to create \sphinxstyleemphasis{Product inventory} as you will click here it will show you

\end{enumerate}

\begin{figure}[htbp]
\centering
\capstart

\noindent\sphinxincludegraphics{{crtnewpinv}.png}
\caption{Creating New Inventory}\label{\detokenize{prodinventory:id10}}\label{\detokenize{prodinventory:id2}}\end{figure}

Here
\begin{quote}
\begin{enumerate}
\def\theenumi{\alph{enumi}}
\def\labelenumi{\theenumi .}
\makeatletter\def\p@enumii{\p@enumi \theenumi .}\makeatother
\item {} 
This is a search field type here \sphinxstyleemphasis{product name} and it will give you suggession select that product if it pops up if not simply type the product name here.

\item {} 
Here set the quantity either by \sphinxstyleemphasis{up and down} arrow which is provided in textfield or Type in this textfield.

\item {} 
Click on the \sphinxstylestrong{Create} button and it will create product inventory.

\end{enumerate}
\begin{enumerate}
\def\theenumi{\arabic{enumi}}
\def\labelenumi{\theenumi .}
\makeatletter\def\p@enumii{\p@enumi \theenumi .}\makeatother
\setcounter{enumi}{3}
\item {} 
Click here to check and manage your reorder, When you click on it It will show like

\end{enumerate}
\end{quote}

\begin{figure}[htbp]
\centering
\capstart

\noindent\sphinxincludegraphics{{reorder}.png}
\caption{Reorder/PO}\label{\detokenize{prodinventory:id11}}\label{\detokenize{prodinventory:id3}}\end{figure}

Here
\begin{enumerate}
\def\theenumi{\alph{enumi}}
\def\labelenumi{\theenumi .}
\makeatletter\def\p@enumii{\p@enumi \theenumi .}\makeatother
\item {} 
Here you can search and select the product by service means from  where you buy it.

\item {} 
Click on pencil button to edit the order as you click it will show like

\end{enumerate}

\begin{figure}[htbp]
\centering
\capstart

\noindent\sphinxincludegraphics{{editreorder}.png}
\caption{Edit Purchase order}\label{\detokenize{prodinventory:id12}}\label{\detokenize{prodinventory:id4}}\end{figure}

Here you have below options
\begin{quote}
\begin{enumerate}
\def\theenumi{\alph{enumi}}
\def\labelenumi{\theenumi .}
\makeatletter\def\p@enumii{\p@enumi \theenumi .}\makeatother
\item {} 
Here you can add products in your purchase list.

\item {} 
Here you can delete the product.

\item {} 
Here you can change the status.

\item {} 
Click on \sphinxstylestrong{Save} button and you are done with Managing Reorder.

\end{enumerate}
\begin{enumerate}
\def\theenumi{\alph{enumi}}
\def\labelenumi{\theenumi .}
\makeatletter\def\p@enumii{\p@enumi \theenumi .}\makeatother
\setcounter{enumi}{2}
\item {} 
Click here it will show you purchase order details for service (From where you buy). It will look like

\end{enumerate}
\end{quote}

\begin{figure}[htbp]
\centering

\noindent\sphinxincludegraphics{{purchaseorderdetails}.png}
\end{figure}

this image.
\begin{enumerate}
\def\theenumi{\arabic{enumi}}
\def\labelenumi{\theenumi .}
\makeatletter\def\p@enumii{\p@enumi \theenumi .}\makeatother
\setcounter{enumi}{4}
\item {} 
Click here it will give you option to download the \sphinxstylestrong{Stock Record} in excel sheet format and it will be  shown through supported application in your computer.

\item {} 
Here you have same options \sphinxstylestrong{Reordering Report} .

\item {} 
Click here to \sphinxstylestrong{Edit Product} , it will take you to below  view.

\end{enumerate}

\begin{figure}[htbp]
\centering

\noindent\sphinxincludegraphics{{editprod}.png}
\end{figure}

If you don,t know how to edit product click here {\hyperref[\detokenize{listing:edit-product}]{\sphinxcrossref{\DUrole{std,std-ref}{Edit Product}}}} .
\begin{enumerate}
\def\theenumi{\arabic{enumi}}
\def\labelenumi{\theenumi .}
\makeatletter\def\p@enumii{\p@enumi \theenumi .}\makeatother
\setcounter{enumi}{7}
\item {} 
Click here to see the product details as you will cilck here it will appear like

\end{enumerate}

\begin{figure}[htbp]
\centering
\capstart

\noindent\sphinxincludegraphics{{proddetails}.png}
\caption{Product status}\label{\detokenize{prodinventory:id13}}\label{\detokenize{prodinventory:id5}}\end{figure}


\subparagraph{Vendor}
\label{\detokenize{prodinventory:vendor}}
\begin{figure}[htbp]
\centering
\capstart

\noindent\sphinxincludegraphics{{vendor}.png}
\caption{Your Vendors}\label{\detokenize{prodinventory:id14}}\label{\detokenize{prodinventory:id6}}\end{figure}

As you can see above this is vendor tab. where you can perform 3 function
\begin{enumerate}
\def\theenumi{\arabic{enumi}}
\def\labelenumi{\theenumi .}
\makeatletter\def\p@enumii{\p@enumi \theenumi .}\makeatother
\item {} 
You can browse \sphinxstylestrong{Vendor} either in the appeared \sphinxstyleemphasis{list} or you can search vendor in \sphinxstyleemphasis{search} bar.

\item {} 
If you are willing to create new vendor click on \sphinxstylestrong{New} and below tab will be open.

\end{enumerate}

\begin{figure}[htbp]
\centering
\capstart

\noindent\sphinxincludegraphics{{crtnewvendor}.png}
\caption{Creating New Vendors}\label{\detokenize{prodinventory:id15}}\label{\detokenize{prodinventory:id7}}\end{figure}

Now let’s start creating new vendor.
\begin{quote}
\begin{quote}
\begin{enumerate}
\def\theenumi{\arabic{enumi}}
\def\labelenumi{\theenumi .}
\makeatletter\def\p@enumii{\p@enumi \theenumi .}\makeatother
\item {} 
Here fill the \sphinxstylestrong{Name} of vendor.

\item {} 
Here fill \sphinxstylestrong{Mobile Number} .

\item {} 
Here enter the \sphinxstylestrong{Email id} of vendor.

\item {} 
Set the \sphinxstylestrong{GST} .

\item {} 
From here onwards keep feeling address of vendor like \sphinxstylestrong{Street} .

\item {} 
City  7. Pin code  8. State and  9. Country

\end{enumerate}
\begin{enumerate}
\def\theenumi{\arabic{enumi}}
\def\labelenumi{\theenumi .}
\makeatletter\def\p@enumii{\p@enumi \theenumi .}\makeatother
\setcounter{enumi}{9}
\item {} 
Here you have two choice if you had filled wrong data about vendor click on \sphinxstylestrong{Reset} else click on \sphinxstylestrong{Save} to create new vendor. Now you will see one new vendor in your vendor list.

\end{enumerate}
\end{quote}
\begin{enumerate}
\def\theenumi{\arabic{enumi}}
\def\labelenumi{\theenumi .}
\makeatletter\def\p@enumii{\p@enumi \theenumi .}\makeatother
\setcounter{enumi}{2}
\item {} 
By clicking on \sphinxstylestrong{vendor profile} you can \sphinxstyleemphasis{edit} vendor details.

\end{enumerate}
\end{quote}

\begin{figure}[htbp]
\centering
\capstart

\noindent\sphinxincludegraphics{{editvendor}.png}
\caption{Editing Vendor’s Info}\label{\detokenize{prodinventory:id16}}\label{\detokenize{prodinventory:id8}}\end{figure}

The above tab is similar to \sphinxstyleemphasis{NEW} tab only differences are there is no \sphinxstyleemphasis{Reset} button and this tab fields are already filled so, you have to just edit some information according to your requirement and click on save so that given information will be updated.


\paragraph{Blog}
\label{\detokenize{blog:blog}}\label{\detokenize{blog:id1}}\label{\detokenize{blog::doc}}
\begin{figure}[htbp]
\centering

\noindent\sphinxincludegraphics{{Blogsetting}.png}
\end{figure}

Above image is showing you a view of blog management Section.
\begin{enumerate}
\def\theenumi{\arabic{enumi}}
\def\labelenumi{\theenumi .}
\makeatletter\def\p@enumii{\p@enumi \theenumi .}\makeatother
\item {} 
Click here to create a new \sphinxstylestrong{Blog} it will open a new tab

\end{enumerate}

\begin{figure}[htbp]
\centering

\noindent\sphinxincludegraphics{{crtnewblog}.png}
\end{figure}

Follow the below step to create a new blog.
\begin{quote}
\begin{enumerate}
\def\theenumi{\alph{enumi}}
\def\labelenumi{\theenumi .}
\makeatletter\def\p@enumii{\p@enumi \theenumi .}\makeatother
\item {} 
This is toggle button you can switch between body and header while writing a blog if you have to write content in header section enable it on header and when you have to write content in body section enable it on body.

\item {} 
Here give the title of blog.

\item {} 
Here seach the tag for that blog and select it accordingly.

\item {} 
From here onwards you have to fill the seo related data here first you enter \sphinxstylestrong{URL suffix} for eg. \sphinxstyleemphasis{.org, .com, .in} accordingly.

\item {} 
Here you set the og image (open graphics image) or

\item {} 
Here you fill the url of OG image.

\item {} 
Here write the \sphinxstylestrong{description} .

\item {} 
Here write the tags for your blog and tags should be seplrated by comma.

\item {} 
Here enter the name of \sphinxstylestrong{section} .

\item {} 
Here enter the \sphinxstylestrong{author name} of particular blog.

\item {} 
Here search the product which you want to attach to this blog just enter the name of product and click on add and your blog is created.

\end{enumerate}
\begin{enumerate}
\def\theenumi{\arabic{enumi}}
\def\labelenumi{\theenumi .}
\makeatletter\def\p@enumii{\p@enumi \theenumi .}\makeatother
\setcounter{enumi}{1}
\item {} 
You can check your blog by entering the \sphinxstylestrong{title} of blog here.

\item {} 
Here you enter the name of category and it will you the particular category’s blog only.

\item {} 
Click here to see \sphinxstylestrong{All Articles} and it will take you back to \sphinxstyleemphasis{My Articles} this single button will help you to switch betweem both.

\item {} 
Click here to see \sphinxstylestrong{Published} and it will take you back to \sphinxstyleemphasis{Drafts} this single button will help you to switch betweem both.

\item {} 
This is a \sphinxstyleemphasis{perveious} button to check the previous page blog.

\item {} 
This is a \sphinxstyleemphasis{next} button to check the previous page blog.

\end{enumerate}
\end{quote}


\paragraph{How to Manage Users}
\label{\detokenize{mng users:how-to-manage-users}}\label{\detokenize{mng users::doc}}
\begin{figure}[htbp]
\centering
\capstart

\noindent\sphinxincludegraphics{{mngusers}.png}
\caption{Manage Users Tab}\label{\detokenize{mng users:id6}}\label{\detokenize{mng users:id1}}\end{figure}

This is the view of manage users tab.
\begin{quote}
\begin{enumerate}
\def\theenumi{\arabic{enumi}}
\def\labelenumi{\theenumi .}
\makeatletter\def\p@enumii{\p@enumi \theenumi .}\makeatother
\item {} 
As you loged in you clicked on manage users it will show the below tab.

\end{enumerate}
\begin{enumerate}
\def\theenumi{\alph{enumi}}
\def\labelenumi{\theenumi .}
\makeatletter\def\p@enumii{\p@enumi \theenumi .}\makeatother
\item {} 
Click on \sphinxstylestrong{New} to create a new user. To create a new user you will get below portal.

\end{enumerate}
\end{quote}

\begin{figure}[htbp]
\centering
\capstart

\noindent\sphinxincludegraphics{{crtnewuser}.png}
\caption{Creating New User}\label{\detokenize{mng users:id7}}\label{\detokenize{mng users:id2}}\end{figure}

Here follow the below steps.
\begin{quote}
\begin{enumerate}
\def\theenumi{\arabic{enumi}}
\def\labelenumi{\theenumi .}
\makeatletter\def\p@enumii{\p@enumi \theenumi .}\makeatother
\item {} 
Enter the \sphinxstylestrong{username}

\item {} 
Here you enter the \sphinxstylestrong{First Name}

\item {} 
Here you enter the \sphinxstylestrong{Last Name}

\item {} 
Here you enter the \sphinxstylestrong{Password}

\item {} 
Here you click on \sphinxstylestrong{save} button and user is created.

\end{enumerate}
\begin{enumerate}
\def\theenumi{\alph{enumi}}
\def\labelenumi{\theenumi .}
\makeatletter\def\p@enumii{\p@enumi \theenumi .}\makeatother
\setcounter{enumi}{1}
\item {} 
Now you are on users tab here is a \sphinxstyleemphasis{search field} click on it and fill the name of user you will see a pop up which cantains user profile.

\item {} 
Click on \sphinxstylestrong{view profile} to check the user details.

\item {} 
Click here if you want to change the user profile (details of user).

\end{enumerate}
\end{quote}

\begin{figure}[htbp]
\centering
\capstart

\noindent\sphinxincludegraphics{{edituserprofile}.png}
\caption{Editing User Details}\label{\detokenize{mng users:id8}}\label{\detokenize{mng users:id3}}\end{figure}

In above portal
\begin{enumerate}
\def\theenumi{\arabic{enumi}}
\def\labelenumi{\theenumi .}
\makeatletter\def\p@enumii{\p@enumi \theenumi .}\makeatother
\item {} 
Here you have to select the \sphinxstylestrong{Prefix}  for user for e.g Miss., Mr., Dr.

\item {} 
Here select the \sphinxstylestrong{gender} of that user.

\item {} 
Here enter the \sphinxstylestrong{City} name.

\item {} 
Here enter the \sphinxstylestrong{Country} name.

\item {} 
Here enter the \sphinxstylestrong{State} name.

\item {} 
Here enter the \sphinxstylestrong{Mobile number} of user.

\item {} 
Here enter the \sphinxstylestrong{Email address} of user.

\item {} 
And click on \sphinxstylestrong{Save} button to save the user details and user profile will be Updated.

\end{enumerate}


\subparagraph{Edit Permision For User}
\label{\detokenize{mng users:edit-permision-for-user}}\begin{enumerate}
\def\theenumi{\alph{enumi}}
\def\labelenumi{\theenumi .}
\makeatletter\def\p@enumii{\p@enumi \theenumi .}\makeatother
\setcounter{enumi}{4}
\item {} 
Here you have two options Click on \sphinxstylestrong{Key icon} to \sphinxstylestrong{Edit permission for User}  and if you have to change the master data of user then click on \sphinxstylestrong{Lock icon} . If you click on key you will get this form to edit the permission for user.

\end{enumerate}

\begin{figure}[htbp]
\centering
\capstart

\noindent\sphinxincludegraphics{{editpermisionforuser}.jpeg}
\caption{Edit Permission for User}\label{\detokenize{mng users:id9}}\label{\detokenize{mng users:id4}}\end{figure}

Here grant the application access and click on \sphinxstylestrong{save} to save the permission.
\begin{quote}

If you click on \sphinxstylestrong{lock} you will get this form to edit master data for user.
\end{quote}

\begin{figure}[htbp]
\centering
\capstart

\noindent\sphinxincludegraphics{{editmasterdataofuser}.png}
\caption{Editing Master Data of User}\label{\detokenize{mng users:id10}}\label{\detokenize{mng users:id5}}\end{figure}

Once the user is registered their account will not be active automatically, To activate that user account Admin has to give permission to that user, for that first open the user profile click on lock icon then you get above tab then
\begin{enumerate}
\def\theenumi{\arabic{enumi}}
\def\labelenumi{\theenumi .}
\makeatletter\def\p@enumii{\p@enumi \theenumi .}\makeatother
\item {} 
Click on \sphinxstylestrong{Acivate} button to activate that user, as the user will be activated the checkbox which is below the \sphinxstyleemphasis{Active ?} will be checked. If the user is staff of that comapny then the staff labeled checkbox will be checked otherwise it will remain unchcked.

\item {} 
Click on \sphinxstylestrong{Generate Passkey} to generate the passkey for that user.

\item {} 
Click on \sphinxstylestrong{Save} button so the generaed passkey will be send to that user and through this passkey user will be able to access their account.

\end{enumerate}


\paragraph{Setting}
\label{\detokenize{globalsetting:setting}}\label{\detokenize{globalsetting::doc}}
\begin{figure}[htbp]
\centering
\capstart

\noindent\sphinxincludegraphics{{globalsetting}.png}
\caption{Configuring Your Website}\label{\detokenize{globalsetting:id4}}\label{\detokenize{globalsetting:id1}}\end{figure}

Now you are on global setting portal you can manage the major part of this site from here.
\begin{enumerate}
\def\theenumi{\arabic{enumi}}
\def\labelenumi{\theenumi .}
\makeatletter\def\p@enumii{\p@enumi \theenumi .}\makeatother
\item {} 
Click here to manage \sphinxstylestrong{Modules And Applications} .

\end{enumerate}

\begin{figure}[htbp]
\centering
\capstart

\noindent\sphinxincludegraphics{{modules&appsetting}.png}
\caption{Setting Modules \& Applications}\label{\detokenize{globalsetting:id5}}\label{\detokenize{globalsetting:id2}}\end{figure}

Above you are seeing the Modules And Application setting portal here
\begin{quote}
\begin{enumerate}
\def\theenumi{\arabic{enumi}}
\def\labelenumi{\theenumi .}
\makeatletter\def\p@enumii{\p@enumi \theenumi .}\makeatother
\item {} 
You can search the modules and application here just enter the name of that modules or application and it will be pop up here.

\item {} 
Click on \sphinxstylestrong{New} to create a new module or applications.

\item {} 
Click here to \sphinxstylestrong{Delete or Edit} that particular module and application.

\item {} 
Click on module to see their description.

\end{enumerate}
\begin{enumerate}
\def\theenumi{\arabic{enumi}}
\def\labelenumi{\theenumi .}
\makeatletter\def\p@enumii{\p@enumi \theenumi .}\makeatother
\setcounter{enumi}{1}
\item {} 
Click here to manage \sphinxstylestrong{Blog} .

\item {} 
Click here to manage \sphinxstylestrong{Ecommerce} .

\end{enumerate}
\end{quote}


\subparagraph{Ecommerce Setting}
\label{\detokenize{globalsetting:ecommerce-setting}}\label{\detokenize{globalsetting:id3}}\begin{quote}

\begin{figure}[htbp]
\centering

\noindent\sphinxincludegraphics{{ecoms1}.jpeg}
\end{figure}
\end{quote}

Now you reached to subportal of \sphinxstyleemphasis{settting} which is \sphinxstylestrong{Ecommerece}, here you will know how easily you can configure your ecommerce setting as per your requirement. Given images is self-explanatory but there is some description too, to make your task easier.
\begin{enumerate}
\def\theenumi{\arabic{enumi}}
\def\labelenumi{\theenumi .}
\makeatletter\def\p@enumii{\p@enumi \theenumi .}\makeatother
\item {} 
In this textarea you will fill your \sphinxstylestrong{service name}.

\item {} 
In this textarea you will fill the URL or font awesome class of the \sphinxstylestrong{logo}.

\item {} 
In this textarea you can fill name of thr \sphinxstylestrong{copyright} holder.

\item {} 
In this textarea you can fill the \sphinxstylestrong{year of copyright}.

\item {} 
Here you paste your \sphinxstylestrong{Facebook page link} (URL string including \sphinxurl{http://} to direct to facebook page).

\end{enumerate}

For your convenience we are explaining part by part.

\begin{figure}[htbp]
\centering

\noindent\sphinxincludegraphics{{ecoms2}.jpeg}
\end{figure}

there is some more awesome features like:
\begin{enumerate}
\def\theenumi{\arabic{enumi}}
\def\labelenumi{\theenumi .}
\makeatletter\def\p@enumii{\p@enumi \theenumi .}\makeatother
\setcounter{enumi}{5}
\item {} 
In this textarea you paste your \sphinxstylestrong{Twitter link} (URL string including \sphinxurl{http://} to direct to twitter page).

\item {} 
In this textarea you paste your \sphinxstylestrong{Linkedin link} (URL string including \sphinxurl{http://} to direct to linkedin page).

\item {} 
In this textarea you paste your \sphinxstylestrong{Playstore link} (URL string including \sphinxurl{http://} to direct to Playstore) to download your Android App.

\item {} 
In the same way in this textarea you paste your \sphinxstylestrong{Appstore} (URL string including \sphinxurl{http://} to direct to Appstore).

\item {} 
In this textarea you fill the color name for your banner text’s bottom color.

\item {} 
Here you fill your \sphinxstylestrong{VAT/TIN Number}.

\end{enumerate}

\begin{figure}[htbp]
\centering

\noindent\sphinxincludegraphics{{ecoms3}.jpeg}
\end{figure}

there are some more awesome features like:
\begin{enumerate}
\def\theenumi{\arabic{enumi}}
\def\labelenumi{\theenumi .}
\makeatletter\def\p@enumii{\p@enumi \theenumi .}\makeatother
\setcounter{enumi}{11}
\item {} 
In this textarea you fill the Global Service Tax Number \sphinxstylestrong{GST NO}.

\item {} 
Here you will fill the \sphinxstylestrong{Address} and you can customise the font size too, just by editing in the \sphinxstyleemphasis{\textless{}font size=’10’\textgreater{}}.

\item {} 
This is checkbox for \sphinxstylestrong{POS scanner}. if you want to enable the POS scanner in your Ecommerce site.

\end{enumerate}

just check \sphinxincludegraphics{{check}.png} the checkbox. If not then,

just uncheck \sphinxincludegraphics{{uncheck}.png} the checkbox.
\begin{quote}
\begin{enumerate}
\def\theenumi{\arabic{enumi}}
\def\labelenumi{\theenumi .}
\makeatletter\def\p@enumii{\p@enumi \theenumi .}\makeatother
\setcounter{enumi}{14}
\item {} 
If you want to enable the \sphinxstylestrong{Rating} feature in your appliation then just check the checkbox if not just leave it unchecked.

\item {} 
If you want to make \sphinxstylestrong{Description visible} then check the checkbox else leave it unchecked.

\item {} 
To provide the \sphinxstylestrong{Filter} option to your customers just check the checkbox else leave it unchecked.

\item {} 
To show the \sphinxstylestrong{Banner Text} on your application just check the checkbox else leave it unchecked.

\item {} 
For \sphinxstylestrong{Multiple Store} just check the checkbox else leave it unchecked.

\end{enumerate}

\begin{figure}[htbp]
\centering

\noindent\sphinxincludegraphics{{ecoms4}.jpeg}
\end{figure}
\end{quote}

lets to some more setting.
\begin{enumerate}
\def\theenumi{\arabic{enumi}}
\def\labelenumi{\theenumi .}
\makeatletter\def\p@enumii{\p@enumi \theenumi .}\makeatother
\setcounter{enumi}{19}
\item {} 
Here you can customise the \sphinxstylestrong{Search} field Textarea.

\item {} 
In this textarea you paste your \sphinxstylestrong{Pinterest link} (URL string including \sphinxurl{http://} to direct to pinterest page).

\item {} 
Here you fill \sphinxstylestrong{Phone Number} of the organisation.

\item {} 
Here you fill \sphinxstylestrong{Email} of the organisation.

\item {} 
Do you want to control the \sphinxstylestrong{Order limit}, for certain reason, just fill this textarea with numbers as many order you want to allow.

\item {} 
Do you want to enable the option to \sphinxstylestrong{Delete POS Product} ? its too easy just check the checkbox else leave it unchecked.

\item {} 
just check this checkbox, it will add one more feature in your application to \sphinxstylestrong{Search} some product for customers.

\end{enumerate}

\begin{figure}[htbp]
\centering

\noindent\sphinxincludegraphics{{ecoms5}.jpeg}
\end{figure}
\begin{description}
\item[{There are some more awesome features which will make your bussiness easier and organised.}] \leavevmode\begin{enumerate}
\def\theenumi{\arabic{enumi}}
\def\labelenumi{\theenumi .}
\makeatletter\def\p@enumii{\p@enumi \theenumi .}\makeatother
\setcounter{enumi}{26}
\item {} 
To enable the \sphinxstylestrong{Cart} option just check the checkbox if you do not want to provide this option to your customer just uncheck it.

\item {} 
Do you want \sphinxstylestrong{Banner Image} on your website ? it is already there, so just check the checkbox. if you don’t want just uncheck this checkbox.

\item {} 
you want to show some more banner on your website to make it beautiful and more informative, just check the checkbox. if you don’t want just uncheck this checkbox.

\item {} 
If you want to show \sphinxstylestrong{maximum category} of your product just check this checkbox, if not, uncheck it then it will show only 5 main category of your products.

\item {} 
If you check this checkbox it will calculate \sphinxstylestrong{GST} (global service tax) too, if not  it will calculate simply.

\item {} 
In this textarea just fill the currency symbole code like \sphinxstylestrong{fa-inr},

\end{enumerate}

\end{description}

it will appear \sphinxincludegraphics{{fa-inr}.png}.
\begin{enumerate}
\def\theenumi{\arabic{enumi}}
\def\labelenumi{\theenumi .}
\makeatletter\def\p@enumii{\p@enumi \theenumi .}\makeatother
\setcounter{enumi}{32}
\item {} 
If you want to show \sphinxstylestrong{Menu on Top} section of website just check this checkbox. if not uncheck it.

\item {} 
This textarea will help to change the font size of \sphinxstylestrong{T\&C} quotation, for this just change the \sphinxstyleemphasis{\textless{}font size=’9’\textgreater{}}.

\end{enumerate}

\begin{figure}[htbp]
\centering

\noindent\sphinxincludegraphics{{ecoms6}.jpeg}
\end{figure}

Some best features are still remaining.
\begin{enumerate}
\def\theenumi{\arabic{enumi}}
\def\labelenumi{\theenumi .}
\makeatletter\def\p@enumii{\p@enumi \theenumi .}\makeatother
\setcounter{enumi}{34}
\item {} 
Here you can customise the fonnt of \sphinxstylestrong{Invoice}.

\item {} 
Here you fill your \sphinxstylestrong{Company Name} and you can customise its font size too just edit \sphinxcode{\sphinxupquote{\textless{}font size='18'\textgreater{}MONOMERCE\textless{}/font\textgreater{}}}.

\item {} 
you can follow the same process for \sphinxstylestrong{Company Address}.

\item {} 
you can follow the same process for \sphinxstylestrong{Company Contact Details} as you do for Company Name.

\item {} 
Same for \sphinxstylestrong{Bank Details} too.

\item {} 
And same for \sphinxstylestrong{Regulatory Details}. in this way you can customise most part of this website.

\end{enumerate}

\begin{figure}[htbp]
\centering

\noindent\sphinxincludegraphics{{ecoms7}.jpeg}
\end{figure}

now we reached at the last section of Ecommerce Setting.
\begin{quote}
\begin{enumerate}
\def\theenumi{\arabic{enumi}}
\def\labelenumi{\theenumi .}
\makeatletter\def\p@enumii{\p@enumi \theenumi .}\makeatother
\setcounter{enumi}{40}
\item {} 
If you want to add the feature of \sphinxstylestrong{Sort by Category} just check this checkbox it will appear as dropdown for category. if you do not want uncheck it.

\item {} 
You want to enable scrolling for category just check the checkbox of \sphinxstylestrong{CategoryScroll} else leave it uncheck.

\item {} 
You want \sphinxstylestrong{Static Banner} (Flag for static banner on top) on your site just check the checkbox else uncheck.

\item {} 
If You want \sphinxstylestrong{Top Icon} (Flag for static banner on top) on your site just check the checkbox else uncheck.

\item {} 
You want \sphinxstylestrong{Cart Service} on your site just check the checkbox else uncheck.

\item {} 
You want \sphinxstylestrong{Cash on Delivery} (COD option on checkout) on your site just check the checkbox else uncheck.

\end{enumerate}
\begin{enumerate}
\def\theenumi{\arabic{enumi}}
\def\labelenumi{\theenumi .}
\makeatletter\def\p@enumii{\p@enumi \theenumi .}\makeatother
\setcounter{enumi}{47}
\item {} 
You want to Enable to see \sphinxstylestrong{Contact Us} in footer just check the checkbox else uncheck.

\item {} 
You want to Enable to see \sphinxstylestrong{Feed Back} in footer just check the checkbox else uncheck.

\item {} 
Now you are almost dne with Ecommerce Setting.

\end{enumerate}
\end{quote}

just click on \sphinxincludegraphics{{save}.png} to save your setting and you will get your customise Frontend.


\paragraph{Admin Profile Options:}
\label{\detokenize{adminprofileoptions:admin-profile-options}}\label{\detokenize{adminprofileoptions:id1}}\label{\detokenize{adminprofileoptions::doc}}
\begin{figure}[htbp]
\centering

\noindent\sphinxincludegraphics{{adminprofileoptions}.png}
\end{figure}

When you click on admin you will see a popup like in above image.
\begin{enumerate}
\def\theenumi{\arabic{enumi}}
\def\labelenumi{\theenumi .}
\makeatletter\def\p@enumii{\p@enumi \theenumi .}\makeatother
\item {} 
Click here to manage \sphinxstylestrong{Admin Stting} .

\end{enumerate}


\subparagraph{Admin Setting}
\label{\detokenize{adminprofileoptions:admin-setting}}\label{\detokenize{adminprofileoptions:id2}}
\begin{figure}[htbp]
\centering

\noindent\sphinxincludegraphics{{adminsetting}.png}
\end{figure}

Here admin can change several things
\begin{quote}
\begin{enumerate}
\def\theenumi{\alph{enumi}}
\def\labelenumi{\theenumi .}
\makeatletter\def\p@enumii{\p@enumi \theenumi .}\makeatother
\item {} 
Here select the \sphinxstylestrong{Display Picture} .

\item {} 
Here admin can change the color of \sphinxstylestrong{main theme} .

\item {} 
Here admin can change the color of \sphinxstylestrong{highlight} color . and in the below form admin can change the password, for this enter the old password in old password textarea, enter the new password in new password textarea, confirm the new password by re-entering the new password in the confirm password textarea.

\item {} 
Click on \sphinxstylestrong{Save} button to update the setting and all the changes will be updated.

\item {} 
Click on \sphinxstyleemphasis{back arrow} symbol to Exit from this tab.

\end{enumerate}
\begin{enumerate}
\def\theenumi{\arabic{enumi}}
\def\labelenumi{\theenumi .}
\makeatletter\def\p@enumii{\p@enumi \theenumi .}\makeatother
\setcounter{enumi}{1}
\item {} 
Click here to see \sphinxstylestrong{About Admin} .

\end{enumerate}
\end{quote}


\subparagraph{About Admin}
\label{\detokenize{adminprofileoptions:about-admin}}\label{\detokenize{adminprofileoptions:id3}}
\begin{figure}[htbp]
\centering

\noindent\sphinxincludegraphics{{aboutadmin}.png}
\end{figure}

When you click on \sphinxstylestrong{About Admin} you will see the above tab.
\begin{enumerate}
\def\theenumi{\arabic{enumi}}
\def\labelenumi{\theenumi .}
\makeatletter\def\p@enumii{\p@enumi \theenumi .}\makeatother
\setcounter{enumi}{2}
\item {} 
Click on \sphinxstylestrong{Logout} to logout from admin side.

\end{enumerate}


\section{FAQS}
\label{\detokenize{faq:faqs}}\label{\detokenize{faq::doc}}
Here some general questions:
\begin{quote}
\begin{enumerate}
\def\theenumi{\arabic{enumi}}
\def\labelenumi{\theenumi .}
\makeatletter\def\p@enumii{\p@enumi \theenumi .}\makeatother
\item {} 
How to register to this website ?

\end{enumerate}

Ans:- {\hyperref[\detokenize{register:registration}]{\sphinxcrossref{\DUrole{std,std-ref}{Registration}}}}
\begin{enumerate}
\def\theenumi{\arabic{enumi}}
\def\labelenumi{\theenumi .}
\makeatletter\def\p@enumii{\p@enumi \theenumi .}\makeatother
\setcounter{enumi}{1}
\item {} 
How to login to this website ?

\end{enumerate}

Ans:- {\hyperref[\detokenize{login:login}]{\sphinxcrossref{\DUrole{std,std-ref}{Login}}}}
\begin{enumerate}
\def\theenumi{\arabic{enumi}}
\def\labelenumi{\theenumi .}
\makeatletter\def\p@enumii{\p@enumi \theenumi .}\makeatother
\setcounter{enumi}{2}
\item {} 
How to Use customer support ?

\end{enumerate}
\end{quote}

Ans:- First you enter in the website, click on \sphinxstyleemphasis{Register} it will take you to registration page there you will get many numeric marked point,
near the point no. 6 you will get \sphinxstylestrong{online} or \sphinxincludegraphics{{werhere}.png}  click on online and you can {\hyperref[\detokenize{register:chat-with-customer-care}]{\sphinxcrossref{\DUrole{std,std-ref}{Chat with customer care}}}} .


\subsection{Some helpful questions \& answers for Admin}
\label{\detokenize{faq:some-helpful-questions-answers-for-admin}}\begin{quote}
\begin{enumerate}
\def\theenumi{\arabic{enumi}}
\def\labelenumi{\theenumi .}
\makeatletter\def\p@enumii{\p@enumi \theenumi .}\makeatother
\item {} 
How to configure the whole website ?

\end{enumerate}

Ans:- Do you want to learn how to configure whole website ? For this first you have to login as admin by clicking on \sphinxstylestrong{Admin login} it will take you to the portal named \sphinxstyleemphasis{Admin loged in} then you have to follow the link {\hyperref[\detokenize{adminlogedin:admin-loged-in}]{\sphinxcrossref{\DUrole{std,std-ref}{admin loged in}}}} .
\begin{enumerate}
\def\theenumi{\arabic{enumi}}
\def\labelenumi{\theenumi .}
\makeatletter\def\p@enumii{\p@enumi \theenumi .}\makeatother
\setcounter{enumi}{1}
\item {} 
How to Control your business through this website ?

\end{enumerate}

Ans:- As per your Question it seems you are admin so, First Login as admin {\hyperref[\detokenize{login:login}]{\sphinxcrossref{\DUrole{std,std-ref}{Login}}}} , Then click on \sphinxstylestrong{Business Management} it will take you to business management portal  and follow the instructions {\hyperref[\detokenize{businessmgmt:business-management}]{\sphinxcrossref{\DUrole{std,std-ref}{Business Management}}}} .
\begin{enumerate}
\def\theenumi{\arabic{enumi}}
\def\labelenumi{\theenumi .}
\makeatletter\def\p@enumii{\p@enumi \theenumi .}\makeatother
\setcounter{enumi}{2}
\item {} 
How to check \sphinxstylestrong{New orders} / \sphinxstylestrong{New Customer} / \sphinxstylestrong{Total Collections} / \sphinxstylestrong{Total sales}  of \sphinxstyleemphasis{Online and Offline Business} ?

\end{enumerate}

Ans:- Click on {\hyperref[\detokenize{businessmgmt:business-management}]{\sphinxcrossref{\DUrole{std,std-ref}{Business Management}}}} and follow the guidelines to find your answer.
\begin{enumerate}
\def\theenumi{\arabic{enumi}}
\def\labelenumi{\theenumi .}
\makeatletter\def\p@enumii{\p@enumi \theenumi .}\makeatother
\setcounter{enumi}{3}
\item {} 
How to login to this website ?

\end{enumerate}

Ans:- {\hyperref[\detokenize{login:login}]{\sphinxcrossref{\DUrole{std,std-ref}{Login}}}}
\begin{enumerate}
\def\theenumi{\arabic{enumi}}
\def\labelenumi{\theenumi .}
\makeatletter\def\p@enumii{\p@enumi \theenumi .}\makeatother
\setcounter{enumi}{4}
\item {} 
How to manage \sphinxstylestrong{Orders} ?

\end{enumerate}

Ans:- {\hyperref[\detokenize{orders:orders}]{\sphinxcrossref{\DUrole{std,std-ref}{Orders}}}}
\begin{enumerate}
\def\theenumi{\arabic{enumi}}
\def\labelenumi{\theenumi .}
\makeatletter\def\p@enumii{\p@enumi \theenumi .}\makeatother
\setcounter{enumi}{5}
\item {} 
How to \sphinxstylestrong{change} or \sphinxstylestrong{check order status} ?

\end{enumerate}

Ans:- Go to Business Management \textgreater{} Business Management E commerce \textgreater{} click on orders and follow {\hyperref[\detokenize{orders:orders}]{\sphinxcrossref{\DUrole{std,std-ref}{Orders}}}} at the point no 7 you can manage the status of orders.
\begin{enumerate}
\def\theenumi{\arabic{enumi}}
\def\labelenumi{\theenumi .}
\makeatletter\def\p@enumii{\p@enumi \theenumi .}\makeatother
\setcounter{enumi}{6}
\item {} 
How the \sphinxstylestrong{Delivery process} work out ?

\end{enumerate}

Ans:- For this you need one app which will be in delivery person side. first he will login to that app
\end{quote}

\begin{figure}[htbp]
\centering

\noindent\sphinxincludegraphics{{dblogin}.png}
\end{figure}

here
\begin{enumerate}
\def\theenumi{\arabic{enumi}}
\def\labelenumi{\theenumi .}
\makeatletter\def\p@enumii{\p@enumi \theenumi .}\makeatother
\item {} 
Here Delivery person will enter the \sphinxstylestrong{Username}.

\item {} 
Here \sphinxstylestrong{Password} and

\item {} 
Click on  \sphinxstylestrong{Login} button. Once he loged in then

\end{enumerate}

\begin{figure}[htbp]
\centering

\noindent\sphinxincludegraphics{{dblogin1}.png}
\end{figure}

Here he will scan the product’s \sphinxstyleemphasis{Bar Code} while receiving the product then product status will be \sphinxstyleemphasis{ongoing} or out for delivery as he reached to customer to deliver the product again he will scan that product and then after the selection of payment mode the particular product status will be change (Ongoing to Delivered) .

\begin{figure}[htbp]
\centering

\noindent\sphinxincludegraphics{{dblogin2}.png}
\end{figure}

here he will select the \sphinxstylestrong{mode of Payment} by clicking on this it will be added in the payment mode list which is currently in background.

\begin{figure}[htbp]
\centering

\noindent\sphinxincludegraphics{{dblogin3}.png}
\end{figure}

Here he can see the order details, how many is \sphinxstyleemphasis{completed} and how many is \sphinxstyleemphasis{ongoing} .
\begin{quote}
\begin{enumerate}
\def\theenumi{\arabic{enumi}}
\def\labelenumi{\theenumi .}
\makeatletter\def\p@enumii{\p@enumi \theenumi .}\makeatother
\item {} 
By sliding right he can check the \sphinxstyleemphasis{previous} date delivery details.

\item {} 
Currently he is on \sphinxstyleemphasis{current} tab. As the delivery person will update here it will reflect in admin side too and admin can manage it from {\hyperref[\detokenize{orders:orders}]{\sphinxcrossref{\DUrole{std,std-ref}{Orders}}}} section.

\end{enumerate}
\begin{enumerate}
\def\theenumi{\arabic{enumi}}
\def\labelenumi{\theenumi .}
\makeatletter\def\p@enumii{\p@enumi \theenumi .}\makeatother
\setcounter{enumi}{7}
\item {} 
How to manage \sphinxstylestrong{online orders} ?

\end{enumerate}
\end{quote}

Ans:- Go to business management \textgreater{} POS you will see the below tab

\begin{figure}[htbp]
\centering

\noindent\sphinxincludegraphics{{posordermng}.png}
\end{figure}

Here
\begin{enumerate}
\def\theenumi{\arabic{enumi}}
\def\labelenumi{\theenumi .}
\makeatletter\def\p@enumii{\p@enumi \theenumi .}\makeatother
\item {} 
Enter customer \sphinxstylestrong{moblie number} and click on search, it will show that customer order details.

\end{enumerate}

\begin{figure}[htbp]
\centering

\noindent\sphinxincludegraphics{{crtanewcustomer}.png}
\end{figure}
\begin{description}
\item[{here}] \leavevmode\begin{enumerate}
\def\theenumi{\arabic{enumi}}
\def\labelenumi{\theenumi .}
\makeatletter\def\p@enumii{\p@enumi \theenumi .}\makeatother
\item {} 
Enter \sphinxstylestrong{User Name} .

\item {} 
Enter \sphinxstylestrong{Mobile Number} .

\item {} 
Here fill the user \sphinxstylestrong{Address} and click on save button it will create a new user in your user list.

\end{enumerate}

\end{description}

Then you will see something like

\begin{figure}[htbp]
\centering

\noindent\sphinxincludegraphics{{clickpay}.png}
\end{figure}

Above tab is showing you the user details and his order, here you can raise the order or delete too according to you requirement. If you do not have to do any change click on \sphinxstylestrong{Pay} . It will take you to payment process.

\begin{figure}[htbp]
\centering

\noindent\sphinxincludegraphics{{payment}.png}
\end{figure}

Now
\begin{quote}
\begin{quote}
\begin{quote}
\begin{enumerate}
\def\theenumi{\arabic{enumi}}
\def\labelenumi{\theenumi .}
\makeatletter\def\p@enumii{\p@enumi \theenumi .}\makeatother
\item {} 
Select the Payment options either cash, card or Wallet (E-Wallet).

\item {} 
If customer is willing to pay on delivery then click on \sphinxstylestrong{Cash on Delivery} .

\item {} 
After completing the payment process click on \sphinxstylestrong{Confirm And Print} button to generate the printed  bill of order.

\end{enumerate}
\begin{enumerate}
\def\theenumi{\arabic{enumi}}
\def\labelenumi{\theenumi .}
\makeatletter\def\p@enumii{\p@enumi \theenumi .}\makeatother
\setcounter{enumi}{1}
\item {} 
Here enter your \sphinxstylestrong{device id} and click on connect and your device will be connected to your portal.

\item {} 
Here fill the order details like \sphinxstyleemphasis{Item description} , \sphinxstyleemphasis{Quantity} , \sphinxstyleemphasis{Rate(Rs.)} and it will show the subtotal in next column.

\item {} 
Click on \sphinxstylestrong{+} button to create new row for more order and fill this row if required (if further order is there).

\item {} 
Here you can \sphinxstylestrong{delete} the particular order.

\item {} 
If you have to delete all the order of that particular user click on \sphinxstylestrong{Reset Form} .

\item {} 
This is back button to go back to previous portal.

\item {} 
Click on pay if payment is done by user

\end{enumerate}
\end{quote}
\begin{enumerate}
\def\theenumi{\arabic{enumi}}
\def\labelenumi{\theenumi .}
\makeatletter\def\p@enumii{\p@enumi \theenumi .}\makeatother
\setcounter{enumi}{8}
\item {} 
How to activate user account ?

\end{enumerate}
\end{quote}

Ans:- follow the path \sphinxstyleemphasis{Login \textgreater{} Admin loged in \textgreater{} How to Manage Users} and then go to {\hyperref[\detokenize{mng users:edit-permision-for-user}]{\sphinxcrossref{\DUrole{std,std-ref}{Edit Permision For User}}}}.
\begin{enumerate}
\def\theenumi{\arabic{enumi}}
\def\labelenumi{\theenumi .}
\makeatletter\def\p@enumii{\p@enumi \theenumi .}\makeatother
\setcounter{enumi}{9}
\item {} 
How to handle \sphinxstylestrong{customer complain} if \sphinxstyleemphasis{transaction is not completed} and order is not placed ?

\end{enumerate}

Ans:- Go to \sphinxstyleemphasis{Business management \textgreater{} Business Management Ecommerce \textgreater{} Orders} then {\hyperref[\detokenize{orders:approve-order}]{\sphinxcrossref{\DUrole{std,std-ref}{Approve Order}}}} .
\begin{enumerate}
\def\theenumi{\arabic{enumi}}
\def\labelenumi{\theenumi .}
\makeatletter\def\p@enumii{\p@enumi \theenumi .}\makeatother
\setcounter{enumi}{10}
\item {} 
How to \sphinxstylestrong{Save Files} ?

\end{enumerate}

Ans:- {\hyperref[\detokenize{adminlogedin:save-files}]{\sphinxcrossref{\DUrole{std,std-ref}{Save Files}}}} .
\begin{enumerate}
\def\theenumi{\arabic{enumi}}
\def\labelenumi{\theenumi .}
\makeatletter\def\p@enumii{\p@enumi \theenumi .}\makeatother
\setcounter{enumi}{11}
\item {} 
How to manage \sphinxstylestrong{Delivery Center} ?

\end{enumerate}

Ans:- To manage \sphinxstyleemphasis{Delivery Center} Click on Delivery Center then you will see the below tab.
\end{quote}

\begin{figure}[htbp]
\centering

\noindent\sphinxincludegraphics{{deliverycenter}.png}
\end{figure}

now
\begin{enumerate}
\def\theenumi{\arabic{enumi}}
\def\labelenumi{\theenumi .}
\makeatletter\def\p@enumii{\p@enumi \theenumi .}\makeatother
\item {} 
Click on \sphinxstylestrong{Load More} to see more \sphinxstyleemphasis{Delivery Center}

\item {} 
Click on that \sphinxstyleemphasis{delivery center} whose order details you want to check then you will see the below tab.

\end{enumerate}

\begin{figure}[htbp]
\centering

\noindent\sphinxincludegraphics{{dcorderdetails}.png}
\end{figure}

Now in header section you can see \sphinxstylestrong{Order Details(16 mins)} it is showing when the order was created.
\begin{enumerate}
\def\theenumi{\arabic{enumi}}
\def\labelenumi{\theenumi .}
\makeatletter\def\p@enumii{\p@enumi \theenumi .}\makeatother
\item {} 
Click on \sphinxstylestrong{Generate Manifest} button to \sphinxstyleemphasis{Generate Manifest} for this order.

\item {} 
If you are feeling the \sphinxstyleemphasis{Order Details} is correct and it is good to go, then click on \sphinxstylestrong{Approve} to approve this order for this it will ask you for confirmation then click on \sphinxstyleemphasis{Yes} button and the order will be approved, in the same way if you have to reject the order the click \sphinxstylestrong{Reject} and give the confirmation by clicking on \sphinxstyleemphasis{Yes} and order will be rejected.

\item {} 
Click on \sphinxstylestrong{Logs \textasciicircum{}} button to write the log for that order then

\item {} 
Write the logs in \sphinxstylestrong{Logs History} text-field and click on \sphinxstylestrong{Submit} button and the logs will be updated for this order.

\item {} 
If you want to \sphinxstyleemphasis{delete} the particular product from this order list then click on the \sphinxstylestrong{Delete} icon of that particular product and then that product will be deleted from list.

\item {} 
If you want to \sphinxstylestrong{add} more product in this order then click on the \sphinxstylestrong{Add +} button.

\end{enumerate}

\begin{figure}[htbp]
\centering

\noindent\sphinxincludegraphics{{addnewproductincurrentorder}.png}
\end{figure}

now
\begin{quote}
\begin{quote}
\begin{enumerate}
\def\theenumi{\alph{enumi}}
\def\labelenumi{\theenumi .}
\makeatletter\def\p@enumii{\p@enumi \theenumi .}\makeatother
\item {} 
Click in the \sphinxstylestrong{Product} text-field and type the product name to search and select product

\item {} 
Enter the quantity of that product in numbers.

\item {} 
Click on \sphinxstylestrong{save} to save and product will be added in that order list.

\end{enumerate}
\begin{enumerate}
\def\theenumi{\arabic{enumi}}
\def\labelenumi{\theenumi .}
\makeatletter\def\p@enumii{\p@enumi \theenumi .}\makeatother
\setcounter{enumi}{12}
\item {} 
How to check \sphinxstylestrong{Reports} \sphinxstyleemphasis{(Sales Reports/Delivery Reports)} ?

\end{enumerate}
\end{quote}

Ans:- To check \sphinxstyleemphasis{sales or delivery reports} click on \sphinxstylestrong{Reports} in Business Management then you will see below tab
\end{quote}

\begin{figure}[htbp]
\centering

\noindent\sphinxincludegraphics{{reports}.png}
\end{figure}

Click on the \sphinxstyleemphasis{Pencil} icon button it will take you to edit that product in below format.
\begin{quote}
\begin{quote}
\begin{enumerate}
\def\theenumi{\arabic{enumi}}
\def\labelenumi{\theenumi .}
\makeatletter\def\p@enumii{\p@enumi \theenumi .}\makeatother
\item {} 
Click on \sphinxstylestrong{Sales Reports} to check \sphinxstyleemphasis{sales reports} .

\item {} 
Click on \sphinxstylestrong{Delivery Reports} to check \sphinxstyleemphasis{delivery reports} .

\item {} 
Select the \sphinxstylestrong{From Date} to \sphinxstylestrong{To Date} to check the reports for that specified days and reports will be appear below.

\item {} 
Click on \sphinxstylestrong{Download} button to \sphinxstyleemphasis{download} the reports in excel sheet format.

\end{enumerate}
\begin{enumerate}
\def\theenumi{\arabic{enumi}}
\def\labelenumi{\theenumi .}
\makeatletter\def\p@enumii{\p@enumi \theenumi .}\makeatother
\setcounter{enumi}{13}
\item {} 
How to \sphinxstylestrong{limit the quantity} of selling products for buyers ?

\end{enumerate}
\end{quote}

Ans:- To limit the quantity goto \sphinxstylestrong{ERP} portal and then click on \sphinxstylestrong{POS} then click on \sphinxstylestrong{Back} button(beside the reset Form button) there you will see the products, hover the cursor over product then you will see below view
\end{quote}

\begin{figure}[htbp]
\centering

\noindent\sphinxincludegraphics{{posproducts}.png}
\end{figure}

Now click on the \sphinxstyleemphasis{Pencil} icon button it will take you to edit that product in below format.

\begin{figure}[htbp]
\centering

\noindent\sphinxincludegraphics{{limitprodquant}.png}
\end{figure}

And set the quantity in \sphinxstylestrong{Order threshold} and that will be the limit.
\begin{quote}
\begin{enumerate}
\def\theenumi{\arabic{enumi}}
\def\labelenumi{\theenumi .}
\makeatletter\def\p@enumii{\p@enumi \theenumi .}\makeatother
\setcounter{enumi}{14}
\item {} 
How to change the image of product with their relative unit size ?

\end{enumerate}

Ans:- which product has multiple pack, it will has a drop-down below price as in below tab.
\end{quote}

\begin{figure}[htbp]
\centering

\noindent\sphinxincludegraphics{{chngprodunits}.png}
\end{figure}

Here if customer selects the different size of pack the display image will change accordingly.
\begin{quote}
\begin{enumerate}
\def\theenumi{\arabic{enumi}}
\def\labelenumi{\theenumi .}
\makeatletter\def\p@enumii{\p@enumi \theenumi .}\makeatother
\setcounter{enumi}{15}
\item {} 
How to set the offer price ?

\end{enumerate}

Ans:- To set the offer price goto \sphinxstylestrong{ERP} portal and then click on \sphinxstylestrong{POS} then click on \sphinxstylestrong{Back} button(beside the reset Form button) there you will see the products, hover the cursor over product then you will see the \sphinxstyleemphasis{Pencil} icon button it will take you to edit that product in below image format.
\end{quote}

\begin{figure}[htbp]
\centering

\noindent\sphinxincludegraphics{{limitprodquant}.png}
\end{figure}

Here set the \sphinxstylestrong{MRP} and \sphinxstylestrong{Selling Price} if selling price will be less then MRP then it will show in offer price. if you set the selling price more than MRP then it will show you message of invalid entry. if MRP and Selling Price will be same it will show as normal price.
\begin{quote}
\begin{enumerate}
\def\theenumi{\arabic{enumi}}
\def\labelenumi{\theenumi .}
\makeatletter\def\p@enumii{\p@enumi \theenumi .}\makeatother
\setcounter{enumi}{16}
\item {} 
How to show products on sale portal(To Customers) ?

\end{enumerate}

Ans:- To show products on sale portal(To Customers) goto \sphinxstylestrong{ERP} portal and then click on \sphinxstylestrong{POS} then click on \sphinxstylestrong{Back} button(beside the reset Form button) there you will see the products, hover the cursor over product then you will see the \sphinxstyleemphasis{Pencil} icon button it will take you to edit that product in above image format. You will see a checkbox labelled by \sphinxstylestrong{Available Online} check that box set the product type and set the quantity for that and click on \sphinxstylestrong{save} button it will be shown to customers.
\begin{enumerate}
\def\theenumi{\arabic{enumi}}
\def\labelenumi{\theenumi .}
\makeatletter\def\p@enumii{\p@enumi \theenumi .}\makeatother
\setcounter{enumi}{17}
\item {} 
How to set offers ?

\end{enumerate}

Ans:- To set offers goto \sphinxstyleemphasis{Edit product} (as you have done for above question) there set the MRP and Selling Price and (in below) \sphinxstylestrong{Special Offer} (Name), \sphinxstylestrong{Description} and click on \sphinxstylestrong{save} button.
\begin{enumerate}
\def\theenumi{\arabic{enumi}}
\def\labelenumi{\theenumi .}
\makeatletter\def\p@enumii{\p@enumi \theenumi .}\makeatother
\setcounter{enumi}{18}
\item {} 
How to add  your service area ?

\end{enumerate}

Ans:- goto {\hyperref[\detokenize{configure:id7}]{\sphinxcrossref{\DUrole{std,std-ref}{Configuring Service Area}}}} and add the pin code of that area.
\end{quote}


\subsection{Some helpful Questions \& answers for Users}
\label{\detokenize{faq:some-helpful-questions-answers-for-users}}\begin{quote}
\begin{enumerate}
\def\theenumi{\arabic{enumi}}
\def\labelenumi{\theenumi .}
\makeatletter\def\p@enumii{\p@enumi \theenumi .}\makeatother
\item {} 
How to \sphinxstylestrong{Login} to this website ?

\end{enumerate}

Ans:- As you entered in the website click on {\hyperref[\detokenize{login:login}]{\sphinxcrossref{\DUrole{std,std-ref}{Login}}}} which will take you to login portal now follow the instruction but ignore point no. 1 because it is only for Admin.
\begin{enumerate}
\def\theenumi{\arabic{enumi}}
\def\labelenumi{\theenumi .}
\makeatletter\def\p@enumii{\p@enumi \theenumi .}\makeatother
\setcounter{enumi}{1}
\item {} 
How to \sphinxstylestrong{Check Products} ?

\end{enumerate}
\end{quote}

Ans:- As you logged in you will get some product categories like Grain, Flours, Whole Spices, Dry Fruits, Rice etc. click on product categories you will see several products related to that category. Now click on that product set the quantity as per your requirement click on add to cart.
\begin{quote}
\begin{enumerate}
\def\theenumi{\arabic{enumi}}
\def\labelenumi{\theenumi .}
\makeatletter\def\p@enumii{\p@enumi \theenumi .}\makeatother
\setcounter{enumi}{2}
\item {} 
How to \sphinxstyleemphasis{Add}  or \sphinxstyleemphasis{Update} \sphinxstylestrong{Shipping Address} ?

\end{enumerate}

Ans:- There is two way to Add your address to this website first is, as you start to order something at that time it will ask you to fill your shipping and billing address and another is Click on your profile for eg. \sphinxstylestrong{ABHISHEK} or whatever your Username is goto \sphinxstylestrong{orders} then click on settings you will get a form like
\end{quote}

\begin{figure}[htbp]
\centering

\noindent\sphinxincludegraphics{{crtnewadd}.png}
\end{figure}

now
\begin{quote}
\begin{enumerate}
\def\theenumi{\alph{enumi}}
\def\labelenumi{\theenumi .}
\makeatletter\def\p@enumii{\p@enumi \theenumi .}\makeatother
\item {} 
Fill the \sphinxstylestrong{Title} of address for e.g. Office, Home etc

\item {} 
Here fill the \sphinxstylestrong{Land Mark} of that address.

\item {} 
Enter the \sphinxstylestrong{Street} number or street name here.

\item {} 
Here enter the pincode of that place then

\item {} 
City f. State g. Country will be auto fillled according to entered pincode if they are not matching you can fill it mannually.

\end{enumerate}
\begin{enumerate}
\def\theenumi{\alph{enumi}}
\def\labelenumi{\theenumi .}
\makeatletter\def\p@enumii{\p@enumi \theenumi .}\makeatother
\setcounter{enumi}{7}
\item {} 
Click on this checkbox to set this address as your \sphinxstyleemphasis{primary address} .

\item {} 
Click on \sphinxstylestrong{Save} button and you address is updated or saved.

\end{enumerate}
\begin{enumerate}
\def\theenumi{\arabic{enumi}}
\def\labelenumi{\theenumi .}
\makeatletter\def\p@enumii{\p@enumi \theenumi .}\makeatother
\setcounter{enumi}{3}
\item {} 
How to \sphinxstylestrong{Order} something ?

\end{enumerate}
\end{quote}

Ans:- To order something first you have to login to this website. then search the product in search field and click on that product, set the quantity and click on add to cart then click on check out if you are login to this website it will take you to product review and then payment options if you are not login then it will take you to login page. if you are still not getting how to order something follow {\hyperref[\detokenize{index:welcome-to-monomerce}]{\sphinxcrossref{\DUrole{std,std-ref}{Welcome to Monomerce}}}} from point number 2.


\section{Quick Guide}
\label{\detokenize{quickguide:quick-guide}}\label{\detokenize{quickguide::doc}}\begin{enumerate}
\def\theenumi{\arabic{enumi}}
\def\labelenumi{\theenumi .}
\makeatletter\def\p@enumii{\p@enumi \theenumi .}\makeatother
\item {} 
{\hyperref[\detokenize{index:id1}]{\sphinxcrossref{\DUrole{std,std-ref}{Welcome to Monomerce}}}}

\item {} 
{\hyperref[\detokenize{index:id2}]{\sphinxcrossref{\DUrole{std,std-ref}{Products View}}}}

\item {} 
{\hyperref[\detokenize{index:id3}]{\sphinxcrossref{\DUrole{std,std-ref}{Add products to cart}}}}

\item {} 
{\hyperref[\detokenize{index:id4}]{\sphinxcrossref{\DUrole{std,std-ref}{Checking out}}}}

\item {} 
{\hyperref[\detokenize{index:id5}]{\sphinxcrossref{\DUrole{std,std-ref}{Products Review}}}}

\item {} 
{\hyperref[\detokenize{index:id6}]{\sphinxcrossref{\DUrole{std,std-ref}{Shipping/Billing Address}}}}

\item {} 
{\hyperref[\detokenize{index:id7}]{\sphinxcrossref{\DUrole{std,std-ref}{Payment Process}}}}

\item {} 
{\hyperref[\detokenize{index:id8}]{\sphinxcrossref{\DUrole{std,std-ref}{Footer Portion of Your website}}}}

\item {} 
{\hyperref[\detokenize{index:id10}]{\sphinxcrossref{\DUrole{std,std-ref}{Cancel/Return Order}}}}

\item {} 
{\hyperref[\detokenize{index:id11}]{\sphinxcrossref{\DUrole{std,std-ref}{User Sending Feedback}}}}

\item {} 
{\hyperref[\detokenize{login:id1}]{\sphinxcrossref{\DUrole{std,std-ref}{Login Page}}}}

\item {} 
{\hyperref[\detokenize{register:id1}]{\sphinxcrossref{\DUrole{std,std-ref}{Registration and Customer Support Page}}}}

\item {} 
{\hyperref[\detokenize{register:id2}]{\sphinxcrossref{\DUrole{std,std-ref}{Chat with customer care}}}}

\item {} 
{\hyperref[\detokenize{custChatoption:id1}]{\sphinxcrossref{\DUrole{std,std-ref}{Customise your chat pannel}}}}

\item {} 
{\hyperref[\detokenize{adminlogedin:id1}]{\sphinxcrossref{\DUrole{std,std-ref}{Admin Portal}}}}

\item {} 
{\hyperref[\detokenize{adminlogedin:id2}]{\sphinxcrossref{\DUrole{std,std-ref}{Saving files}}}}

\item {} 
{\hyperref[\detokenize{businessmgmt:id1}]{\sphinxcrossref{\DUrole{std,std-ref}{Your Business Report}}}}

\item {} 
{\hyperref[\detokenize{listing:id1}]{\sphinxcrossref{\DUrole{std,std-ref}{Listing Products}}}}

\item {} 
{\hyperref[\detokenize{listing:id2}]{\sphinxcrossref{\DUrole{std,std-ref}{Editing And Creating Products}}}}

\item {} 
{\hyperref[\detokenize{listing:id3}]{\sphinxcrossref{\DUrole{std,std-ref}{Editing Products}}}}

\item {} 
{\hyperref[\detokenize{listing:id4}]{\sphinxcrossref{\DUrole{std,std-ref}{Creating List}}}}

\item {} 
{\hyperref[\detokenize{listing:id5}]{\sphinxcrossref{\DUrole{std,std-ref}{Products list}}}}

\item {} 
{\hyperref[\detokenize{listing:id6}]{\sphinxcrossref{\DUrole{std,std-ref}{Bulk List creation}}}}

\item {} 
{\hyperref[\detokenize{configure:id1}]{\sphinxcrossref{\DUrole{std,std-ref}{Configuring your website}}}}

\item {} 
{\hyperref[\detokenize{configure:id2}]{\sphinxcrossref{\DUrole{std,std-ref}{Managing Products}}}}

\item {} 
{\hyperref[\detokenize{configure:id3}]{\sphinxcrossref{\DUrole{std,std-ref}{Edit Genric Products}}}}

\item {} 
{\hyperref[\detokenize{configure:id4}]{\sphinxcrossref{\DUrole{std,std-ref}{Creating Offer Banner}}}}

\item {} 
{\hyperref[\detokenize{configure:id5}]{\sphinxcrossref{\DUrole{std,std-ref}{Creating Promocode}}}}

\item {} 
{\hyperref[\detokenize{configure:id6}]{\sphinxcrossref{\DUrole{std,std-ref}{Frequently Asked questions and answers about Your Website.}}}}

\item {} 
{\hyperref[\detokenize{configure:id8}]{\sphinxcrossref{\DUrole{std,std-ref}{Setting Images for your website}}}}

\item {} 
{\hyperref[\detokenize{orders:id1}]{\sphinxcrossref{\DUrole{std,std-ref}{Orders}}}}

\item {} 
{\hyperref[\detokenize{orders:id2}]{\sphinxcrossref{\DUrole{std,std-ref}{Approving Orders}}}}

\item {} 
{\hyperref[\detokenize{orders:id3}]{\sphinxcrossref{\DUrole{std,std-ref}{Generating menifest}}}}

\item {} 
{\hyperref[\detokenize{support:id1}]{\sphinxcrossref{\DUrole{std,std-ref}{Support}}}}

\item {} 
{\hyperref[\detokenize{pages:id1}]{\sphinxcrossref{\DUrole{std,std-ref}{Creating several types of pages for your business.}}}}

\item {} 
{\hyperref[\detokenize{BMpos:id1}]{\sphinxcrossref{\DUrole{std,std-ref}{POS (Point of sale)}}}}

\item {} 
{\hyperref[\detokenize{BMpos:id2}]{\sphinxcrossref{\DUrole{std,std-ref}{Customers}}}}

\item {} 
{\hyperref[\detokenize{BMpos:id3}]{\sphinxcrossref{\DUrole{std,std-ref}{Invoices}}}}

\item {} 
{\hyperref[\detokenize{prodinventory:id1}]{\sphinxcrossref{\DUrole{std,std-ref}{Product inventory}}}}

\item {} 
{\hyperref[\detokenize{prodinventory:id2}]{\sphinxcrossref{\DUrole{std,std-ref}{Creating New Inventory}}}}

\item {} 
{\hyperref[\detokenize{prodinventory:id3}]{\sphinxcrossref{\DUrole{std,std-ref}{Reorder/PO}}}}

\item {} 
{\hyperref[\detokenize{prodinventory:id4}]{\sphinxcrossref{\DUrole{std,std-ref}{Edit Purchase order}}}}

\item {} 
{\hyperref[\detokenize{prodinventory:id5}]{\sphinxcrossref{\DUrole{std,std-ref}{Product status}}}}

\item {} 
{\hyperref[\detokenize{prodinventory:id6}]{\sphinxcrossref{\DUrole{std,std-ref}{Your Vendors}}}}

\item {} 
{\hyperref[\detokenize{prodinventory:id7}]{\sphinxcrossref{\DUrole{std,std-ref}{Creating New Vendors}}}}

\item {} 
{\hyperref[\detokenize{prodinventory:id8}]{\sphinxcrossref{\DUrole{std,std-ref}{Editing Vendor’s Info}}}}

\item {} 
{\hyperref[\detokenize{blog:id1}]{\sphinxcrossref{\DUrole{std,std-ref}{Blog}}}}

\item {} 
{\hyperref[\detokenize{mng users:id1}]{\sphinxcrossref{\DUrole{std,std-ref}{Manage Users Tab}}}}

\item {} 
{\hyperref[\detokenize{mng users:id2}]{\sphinxcrossref{\DUrole{std,std-ref}{Creating New User}}}}

\item {} 
{\hyperref[\detokenize{mng users:id3}]{\sphinxcrossref{\DUrole{std,std-ref}{Editing User Details}}}}

\item {} 
{\hyperref[\detokenize{mng users:id4}]{\sphinxcrossref{\DUrole{std,std-ref}{Edit Permission for User}}}}

\item {} 
{\hyperref[\detokenize{mng users:id5}]{\sphinxcrossref{\DUrole{std,std-ref}{Editing Master Data of User}}}}

\item {} 
{\hyperref[\detokenize{globalsetting:id1}]{\sphinxcrossref{\DUrole{std,std-ref}{Configuring Your Website}}}}

\item {} 
{\hyperref[\detokenize{globalsetting:id2}]{\sphinxcrossref{\DUrole{std,std-ref}{Setting Modules \& Applications}}}}

\item {} 
{\hyperref[\detokenize{globalsetting:id3}]{\sphinxcrossref{\DUrole{std,std-ref}{Ecommerce Setting}}}}

\item {} 
{\hyperref[\detokenize{adminprofileoptions:id1}]{\sphinxcrossref{\DUrole{std,std-ref}{Admin Profile Options:}}}}

\item {} 
{\hyperref[\detokenize{adminprofileoptions:id2}]{\sphinxcrossref{\DUrole{std,std-ref}{Admin Setting}}}}

\item {} 
{\hyperref[\detokenize{adminprofileoptions:id3}]{\sphinxcrossref{\DUrole{std,std-ref}{About Admin}}}}

\end{enumerate}



\renewcommand{\indexname}{Index}
\printindex
\end{document}