%% Generated by Sphinx.
\def\sphinxdocclass{report}
\documentclass[letterpaper,10pt,english]{sphinxmanual}
\ifdefined\pdfpxdimen
   \let\sphinxpxdimen\pdfpxdimen\else\newdimen\sphinxpxdimen
\fi \sphinxpxdimen=.75bp\relax

\usepackage[utf8]{inputenc}
\ifdefined\DeclareUnicodeCharacter
 \ifdefined\DeclareUnicodeCharacterAsOptional
  \DeclareUnicodeCharacter{"00A0}{\nobreakspace}
  \DeclareUnicodeCharacter{"2500}{\sphinxunichar{2500}}
  \DeclareUnicodeCharacter{"2502}{\sphinxunichar{2502}}
  \DeclareUnicodeCharacter{"2514}{\sphinxunichar{2514}}
  \DeclareUnicodeCharacter{"251C}{\sphinxunichar{251C}}
  \DeclareUnicodeCharacter{"2572}{\textbackslash}
 \else
  \DeclareUnicodeCharacter{00A0}{\nobreakspace}
  \DeclareUnicodeCharacter{2500}{\sphinxunichar{2500}}
  \DeclareUnicodeCharacter{2502}{\sphinxunichar{2502}}
  \DeclareUnicodeCharacter{2514}{\sphinxunichar{2514}}
  \DeclareUnicodeCharacter{251C}{\sphinxunichar{251C}}
  \DeclareUnicodeCharacter{2572}{\textbackslash}
 \fi
\fi
\usepackage{cmap}
\usepackage[T1]{fontenc}
\usepackage{amsmath,amssymb,amstext}
\usepackage{babel}
\usepackage{times}
\usepackage[Bjarne]{fncychap}
\usepackage[dontkeepoldnames]{sphinx}

\usepackage{geometry}

% Include hyperref last.
\usepackage{hyperref}
% Fix anchor placement for figures with captions.
\usepackage{hypcap}% it must be loaded after hyperref.
% Set up styles of URL: it should be placed after hyperref.
\urlstyle{same}
\addto\captionsenglish{\renewcommand{\contentsname}{Modules:}}

\addto\captionsenglish{\renewcommand{\figurename}{Fig.}}
\addto\captionsenglish{\renewcommand{\tablename}{Table}}
\addto\captionsenglish{\renewcommand{\literalblockname}{Listing}}

\addto\captionsenglish{\renewcommand{\literalblockcontinuedname}{continued from previous page}}
\addto\captionsenglish{\renewcommand{\literalblockcontinuesname}{continues on next page}}

\addto\extrasenglish{\def\pageautorefname{page}}

\setcounter{tocdepth}{1}



\title{Burderer Documentation}
\date{Jan 31, 2019}
\release{0.0.1}
\author{aj}
\newcommand{\sphinxlogo}{\vbox{}}
\renewcommand{\releasename}{Release}
\makeindex

\begin{document}

\maketitle
\sphinxtableofcontents
\phantomsection\label{\detokenize{index::doc}}


\begin{figure}[htbp]
\centering
\capstart

\noindent\sphinxincludegraphics{{loginpage}.png}
\caption{Login Page}\label{\detokenize{index:id1}}\end{figure}

Here you enter the \sphinxstylestrong{Username} and \sphinxstylestrong{Password} .


\chapter{Manage Users}
\label{\detokenize{manageuser::doc}}\label{\detokenize{manageuser:welcome-to-burderer-s-documentation}}\label{\detokenize{manageuser:manage-users}}
\begin{figure}[htbp]
\centering

\noindent\sphinxincludegraphics{{mnguser}.png}
\end{figure}

This is the portal to manage users, how? lets check it.
\begin{enumerate}
\item {} 
Click here to create a \sphinxstylestrong{new} user.

\end{enumerate}

\begin{figure}[htbp]
\centering

\noindent\sphinxincludegraphics{{crtnewuser}.png}
\end{figure}

Creating new user is very simple just follow the below steps.
\begin{quote}
\begin{enumerate}
\item {} 
Fill the \sphinxstylestrong{username} in this text-area.

\item {} 
Here you fill your \sphinxstylestrong{first name}.

\item {} 
Here you fill your \sphinxstylestrong{last name}.

\item {} 
Create your \sphinxstylestrong{password} (just fill in this text area) .

\item {} 
Click on \sphinxstylestrong{save} button and a new user is created.

\end{enumerate}
\begin{enumerate}
\setcounter{enumi}{1}
\item {} 
To \sphinxstylestrong{search} user click here this is search tab. This is default page of  {\hyperref[\detokenize{manageuser:manage-users}]{\sphinxcrossref{Manage Users}}} .

\item {} 
These are buttons to change these pages  by clicking on numbers or by clicking on \sphinxstylestrong{next} and \sphinxstylestrong{previous} button which is beside the \sphinxstyleemphasis{number} buttons.

\item {} 
This is \sphinxstylestrong{search} field. you want to check user profile \sphinxstyleemphasis{search} them here by \sphinxstyleemphasis{username} .

\end{enumerate}
\end{quote}

\begin{figure}[htbp]
\centering

\noindent\sphinxincludegraphics{{searchuser}.png}
\end{figure}

In above image your search result will be appear.
\begin{enumerate}
\setcounter{enumi}{4}
\item {} 
Click here it will open user profile where you can \sphinxstylestrong{edit} it.

\end{enumerate}

\begin{figure}[htbp]
\centering

\noindent\sphinxincludegraphics{{edituser}.png}
\end{figure}

Here you can edit user profile.
\begin{quote}
\begin{enumerate}
\item {} 
Here you enter the \sphinxstylestrong{user id (Employee id)}  .

\item {} 
Here select the \sphinxstylestrong{prefix} .

\item {} 
Here select the \sphinxstylestrong{gender} .

\item {} 
Here \sphinxstylestrong{Browse image} file for display picture.

\item {} 
Click on \sphinxstylestrong{save} button to save the profile.

\end{enumerate}
\begin{enumerate}
\setcounter{enumi}{5}
\item {} 
Click on this \sphinxstylestrong{key button} to \sphinxstyleemphasis{Edit the permission for that user} . As you click, you will see the below tab.

\end{enumerate}
\end{quote}

\begin{figure}[htbp]
\centering

\noindent\sphinxincludegraphics{{editpermissionforuser}.png}
\end{figure}

In the above tab
\begin{quote}
\begin{enumerate}
\item {} 
Search and \sphinxstylestrong{Select Role} for that user. for e.g Manager

\item {} 
Here search and select the Access area for user, You can give permission to as many field as you want.

\item {} 
Then click on \sphinxstylestrong{save} button and that user will be able to access only those areas which is granted by Admin.

\end{enumerate}
\begin{enumerate}
\setcounter{enumi}{6}
\item {} 
Click on \sphinxstylestrong{Lock} button to \sphinxstyleemphasis{edit the master data for that user} .

\end{enumerate}
\end{quote}

\begin{figure}[htbp]
\centering

\noindent\sphinxincludegraphics{{editmasterdataforuser}.png}
\end{figure}

Let’s see how ?
\begin{quote}
\begin{enumerate}
\item {} 
Here you will see the \sphinxstylestrong{username} which is not editable.

\item {} 
Here you can edit user’s \sphinxstylestrong{First Name} .

\item {} 
Here you can edit user’s \sphinxstylestrong{Last Name} .

\item {} 
Here you can edit user’s \sphinxstylestrong{Password} as well.

\item {} 
Here click on this check box to \sphinxstylestrong{Active} that user. if this checkbox is checked means user is already active if not admin can make him as a active user.

\item {} 
If that user is a \sphinxstylestrong{staff} member then this checkbox will be marked as checked or if you want to add them as you staff member then click on this and make this checked.

\item {} 
Click on \sphinxstylestrong{Save} button and the edited data will be saved.

\end{enumerate}
\begin{enumerate}
\setcounter{enumi}{7}
\item {} 
Click on \sphinxstylestrong{View  Profile} to check user’s details.

\end{enumerate}
\end{quote}


\chapter{Projects}
\label{\detokenize{projects::doc}}\label{\detokenize{projects:projects}}
\begin{figure}[htbp]
\centering

\noindent\sphinxincludegraphics{{projects}.png}
\end{figure}

This is \sphinxstylestrong{projects} portal you can call it products as well. Do you want to create new project ?
\begin{quote}
\begin{enumerate}
\item {} 
Click on \sphinxstylestrong{New} and just follow the below given steps and you will see a new project (PO) in your inventory.

\end{enumerate}
\begin{enumerate}
\item {} 
Search and select the name of \sphinxstylestrong{Vendor} in this text area, it will be appear according to your vendor list from \sphinxstyleemphasis{vendor} portal means that here you can search only those vendor which you had created from your vendor portal.

\item {} 
Fill \sphinxstylestrong{Customer Name} here if that customer is already in your list then it will come as popup and you can select it else you can add them as your customer by clicking on \sphinxstylestrong{new} button.

\end{enumerate}
\end{quote}

\begin{sphinxadmonition}{note}{Note:}
New button appears beside Customer name, only when the new customer entry happens .
\end{sphinxadmonition}

\begin{figure}[htbp]
\centering

\noindent\sphinxincludegraphics{{createnewservice(customer)}.png}
\end{figure}

In the above tab fill the customer details, it is optional but if you want to keep detail of this customer
\begin{quote}
\begin{quote}
\begin{enumerate}
\item {} 
Enter \sphinxstylestrong{company} name here.

\item {} 
Enter customer \sphinxstylestrong{name} .

\item {} 
Enter customer’s \sphinxstylestrong{Email} here.

\item {} 
Here enter the \sphinxstylestrong{gst \%} .

\item {} 
Here enter the mobile number and address in below form and

\item {} 
Click on \sphinxstylestrong{Save Company Details} to save these above details.

\end{enumerate}
\begin{enumerate}
\setcounter{enumi}{2}
\item {} 
Give the \sphinxstylestrong{Title} to your project.

\item {} 
Search and select the \sphinxstylestrong{Responsible User} name for this project and click on Add button to \sphinxstylestrong{ADD} him/her to this project.

\item {} 
Set the \sphinxstylestrong{Tentative Closing Date} for this project, either by selecting through \sphinxstyleemphasis{Date Picker} (which is just beside the text area) or just fill the date in this text area in DD-MM-YYYY format.

\item {} 
Here fill the \sphinxstylestrong{Machine Model} , it means product identification number.

\item {} 
Here fill the \sphinxstylestrong{Comm nr} (commission number).

\item {} 
Here enter the \sphinxstylestrong{Quote Reference} .

\item {} 
Here enter the \sphinxstylestrong{Inquiry Reference} .

\item {} 
Click on \sphinxstylestrong{Reset} if you want \sphinxstyleemphasis{to change} this project information else save it by clicking on \sphinxstylestrong{Save} button.

\end{enumerate}
\end{quote}
\begin{enumerate}
\setcounter{enumi}{1}
\item {} 
Click on \sphinxstylestrong{Archive} to see your archived projects.

\item {} 
Once you moved your project in \sphinxstylestrong{junk} it will be deleted after some period of time.

\item {} 
This is one of commission numbers you can check it’s details by clicking on it.

\end{enumerate}
\end{quote}

\begin{figure}[htbp]
\centering

\noindent\sphinxincludegraphics{{commno}.png}
\end{figure}

In the above tab you can see
\begin{enumerate}
\item {} 
Here you see \sphinxstylestrong{Total ordered and Total consumed graph} .

\item {} 
Here you see only that particular \sphinxstylestrong{project details} .

\end{enumerate}

\begin{figure}[htbp]
\centering

\noindent\sphinxincludegraphics{{projectdetails1}.png}
\end{figure}

Here you can see elaborated details of that particular project like..
\begin{enumerate}
\item {} 
Here you can see the \sphinxstylestrong{Project Title} , \sphinxstylestrong{Customer name} , \sphinxstylestrong{Closing Date} , \sphinxstylestrong{Machine Model} , \sphinxstylestrong{Comm Nr.} , \sphinxstylestrong{Quote Reference} , \sphinxstylestrong{Quote Reference} , \sphinxstylestrong{Enquiry Reference} and can change as well.

\item {} 
Here you can see the Name of \sphinxstylestrong{Responsible Person} and their \sphinxstylestrong{ID} .

\item {} 
Here you can see the \sphinxstylestrong{Revision} number of that project.

\item {} 
Here you can select the \sphinxstylestrong{Currency Type} like \sphinxstylestrong{CHF, INR, EUR, USD, JPY, GBP, AUD, CAD, ZAR} ( we can addsome more currency type according to you demand) and according to that the invoice and price will be generated.

\item {} 
Here enter the \sphinxstylestrong{PO Number} (Purchases Order Number).

\item {} 
Here you can see and change the \sphinxstylestrong{PO Date} for this project, either by selecting through \sphinxstyleemphasis{Date Picker} (which is just beside the text area) or just fill the date in this text area in DD-MM-YYYY format.

\item {} 
Here you can see and change the \sphinxstylestrong{Quotation Number} .

\item {} 
Here you can see and change the \sphinxstylestrong{Quotation Date} for this project, either by selecting through \sphinxstyleemphasis{Date Picker} (which is just beside the text area) or just fill the date in this text area in DD-MM-YYYY format.

\item {} 
Here you can see and change the \sphinxstylestrong{Invoice Number} .

\item {} 
Here you can see and change the \sphinxstylestrong{BOE Number} (Bill of exchange number).

\item {} 
Here you can see and change the several values like \sphinxstylestrong{Ex Work Price(can not change from here)} , \sphinxstylestrong{Packing} , \sphinxstylestrong{Insurance} , \sphinxstylestrong{Freight} , \sphinxstylestrong{Assemble Value} , \sphinxstylestrong{Gst} and \sphinxstylestrong{Clearing charges} .

\end{enumerate}

\begin{figure}[htbp]
\centering

\noindent\sphinxincludegraphics{{projectdetails2}.png}
\end{figure}

And
\begin{enumerate}
\setcounter{enumi}{11}
\item {} 
Here you can see the \sphinxstylestrong{Net Weight} .

\item {} 
Here you can see and change the \sphinxstylestrong{Ex Rate} .

\item {} 
Here you can see and change the \sphinxstylestrong{Profit Margin} .

\item {} 
Here you can \sphinxstylestrong{Download PO} in your selected currency type which you had selected in above at point no 4.

\item {} 
Here you can \sphinxstylestrong{Download PO} in INR means in the terms of Indian currency.

\item {} 
Just click here to \sphinxstylestrong{Download landing Details} .

\item {} 
Here you can \sphinxstylestrong{Download Quotation} in your selected currency type which you had selected in above at point no 4.

\item {} 
Here you can \sphinxstylestrong{Download Quotation} in INR means in the terms of Indian currency.

\item {} 
Here in this table you can see and change product details.

\item {} 
By clicking on delete button you can \sphinxstylestrong{Delete} that particular row from the table.

\item {} 
After this click on \sphinxstylestrong{Send For Approval} button and it will be send for approval.

\end{enumerate}

\begin{figure}[htbp]
\centering

\noindent\sphinxincludegraphics{{projectdetails3}.png}
\end{figure}

And you will see the above tab here
\begin{enumerate}
\item {} 
if You are OK with project details click on \sphinxstylestrong{Approve} button it will ask you for confirmation click on yes to approve.

\item {} 
if You are not OK with project details click on \sphinxstylestrong{Reject} button it will ask you for confirmation click on yes to reject. in the next tab click on \sphinxstylestrong{Save} it will ask you for confirmation click on yes and product will be added into your inventory.

\end{enumerate}

\begin{sphinxadmonition}{note}{Note:}
you will see that the \sphinxstylestrong{Save} button will be transform into \sphinxstylestrong{Archive} button.
\end{sphinxadmonition}

\begin{figure}[htbp]
\centering

\noindent\sphinxincludegraphics{{inventory}.png}
\end{figure}

Above is added product in inventory.
\begin{enumerate}
\setcounter{enumi}{2}
\item {} 
Here you see the full details in tabular form for that commission number.

\item {} 
By clicking on \sphinxstylestrong{pencil} button you can edit that project details which is similar to creating project only difference is here the details are already filled you have to change these value and click on save and that project details will be updated.

\item {} 
By clicking on \sphinxstylestrong{Delete} button you can delete the project.

\end{enumerate}

\begin{sphinxadmonition}{note}{Note:}
For one commission number there can be many projects.
\end{sphinxadmonition}


\chapter{Master sheet}
\label{\detokenize{mastersheet:master-sheet}}\label{\detokenize{mastersheet::doc}}
\begin{figure}[htbp]
\centering

\noindent\sphinxincludegraphics{{mastersheet}.png}
\end{figure}

Here you can check all the products and their detail information.
\begin{enumerate}
\item {} 
Click on new to add a new product in master sheet.

\end{enumerate}

\begin{figure}[htbp]
\centering

\noindent\sphinxincludegraphics{{addnewprodinmastersheet}.png}
\end{figure}

In the above tab you have to fill the below details.
\begin{quote}
\begin{enumerate}
\item {} 
Enter the \sphinxstylestrong{part number} of that product which you want to add in master sheet.

\item {} 
Here enter the \sphinxstylestrong{description} for the product main description enter in \sphinxstyleemphasis{Description 1}

\item {} 
Here you can enter the detail description about that product \sphinxstyleemphasis{Description 2} will let you know the slight difference between two similar (same named) product.

\item {} 
suppose you were having some product but know you found their alternate so you can keep that info too about that product, for this you have to fill the \sphinxstylestrong{Replace By} field with the older product name.

\item {} 
Here \sphinxstylestrong{enter} the weight of that product.

\item {} 
Here you will fill (set) the \sphinxstylestrong{price} for that product.

\item {} 
Here enter the  \sphinxstylestrong{customs number} .

\item {} 
Here enter the \sphinxstylestrong{GST \%} .

\item {} 
Here enter the \sphinxstylestrong{Custom \%} .

\item {} 
Here enter the \sphinxstylestrong{Bar code} of that product .

\item {} 
and after filling the product details click on \sphinxstylestrong{Create} button and product will reflect in Master sheet as per your creation.

\end{enumerate}
\begin{enumerate}
\setcounter{enumi}{1}
\item {} 
To upload products in bulk amount click on \sphinxstylestrong{Upload Button}

\end{enumerate}
\end{quote}

\begin{figure}[htbp]
\centering

\noindent\sphinxincludegraphics{{uploadproductsheet}.png}
\end{figure}

and \sphinxstylestrong{Choose} the excel sheet from your computer which is containing products details and click on \sphinxstylestrong{Upload} button and these all products which was save in excel sheet will come as the output in master sheet.
\begin{enumerate}
\setcounter{enumi}{2}
\item {} 
This is the \sphinxstylestrong{search} field to search the product in master sheet here you can \sphinxstyleemphasis{search product by part number} .

\item {} 
By clicking on \sphinxstylestrong{pencil} button you can edit that product details which is similar to adding a new product in master sheet only difference is here the details are already filled you have to change these value and click on save and that product  details will be updated in master sheet.

\item {} 
By clicking on \sphinxstylestrong{delete} button you can delete that product from master sheet.

\end{enumerate}


\chapter{Inventory}
\label{\detokenize{inventory::doc}}\label{\detokenize{inventory:inventory}}
\begin{figure}[htbp]
\centering

\noindent\sphinxincludegraphics{{inventory1}.png}
\end{figure}

Above image shows the Inventory tab. Let’s explore it.
\begin{enumerate}
\item {} 
This is a toggle switch which will allow to switch between two tabs first is \sphinxstylestrong{Inventory} and another is \sphinxstylestrong{Material Issue} . In above image it is showing the inventory tab, just click on it, it will take you to \sphinxstyleemphasis{Material issue} .

\item {} 
This is a \sphinxstylestrong{search} field here you can \sphinxstyleemphasis{search} products.

\item {} 
Just right of it there is refresh button to \sphinxstyleemphasis{refresh} your opened pages and at the same place there is \sphinxstyleemphasis{next} and \sphinxstyleemphasis{previous} button too.

\item {} 
Click on \sphinxstylestrong{Add to Cart} button to for your customer.

\item {} 
Your added products number will be shown here (at cart icon) and click here .

\end{enumerate}

\begin{figure}[htbp]
\centering

\noindent\sphinxincludegraphics{{cart}.png}
\end{figure}

Now, here you can manage cart too.
\begin{quote}
\begin{enumerate}
\item {} 
Here fill \sphinxstylestrong{Project name} .

\item {} 
Mention the \sphinxstylestrong{Engineer} name here.

\item {} 
Here you can manage the \sphinxstylestrong{Quantity} of particular added product just bring your cursor into that product’s Quantity column and change it according to your wish just by entering the value or you can change it by clicking on \sphinxstyleemphasis{Up} and \sphinxstyleemphasis{Down} arrow too.

\item {} 
If you want to delete the product then click on \sphinxstylestrong{Delete} icon.

\item {} 
Now click on \sphinxstylestrong{Save} button to save the cart’s products.

\end{enumerate}
\begin{enumerate}
\setcounter{enumi}{5}
\item {} 
Click on new to create a \sphinxstylestrong{New} product in inventory.

\end{enumerate}
\end{quote}

\begin{figure}[htbp]
\centering

\noindent\sphinxincludegraphics{{crtaproduct}.png}
\end{figure}

Now to create a product
\begin{quote}
\begin{enumerate}
\item {} 
Here fill the \sphinxstylestrong{Product Name} .

\item {} 
Here set the \sphinxstylestrong{Quantity} of that product.

\item {} 
Here set the \sphinxstylestrong{Price} of that product.

\item {} 
Now click on \sphinxstylestrong{Save} and product is created for inventory.

\end{enumerate}
\begin{enumerate}
\setcounter{enumi}{6}
\item {} 
Click on \sphinxstylestrong{Create Stock Summary} and the details will be saved in stock which you can check by clicking on \sphinxstyleemphasis{Stock Check} .

\item {} 
Click on \sphinxstylestrong{Download Stock Summary} and it will download a excel sheet in your system which will contain the stock information.

\item {} 
Click on \sphinxstylestrong{Product Stock Consumed} to check the how much product is consumed from stock.

\item {} 
Click on \sphinxstylestrong{Invoice BOE} and download the excel sheet and it will show you all the Invoice BOE.

\item {} 
Click on \sphinxstylestrong{Stock Detail} and it will show you stock detail in the PDF format.

\item {} 
Click on \sphinxstylestrong{Reset} button to reset the inventory.

\item {} 
Click on \sphinxstyleemphasis{product} which will show you the product details like Quantity, Price and Total.

\end{enumerate}
\end{quote}

Now lets click on the toggle button it will show you the material issued list.

\begin{figure}[htbp]
\centering

\noindent\sphinxincludegraphics{{materiaissued}.png}
\end{figure}

From above tab
\begin{enumerate}
\item {} 
Click on \sphinxstylestrong{Stock Check} button to and it will show you the list like give below.

\end{enumerate}

\begin{figure}[htbp]
\centering

\noindent\sphinxincludegraphics{{stockcheck}.png}
\end{figure}

From above tab you can change the stock check list by clicking on \sphinxstylestrong{Prev} and \sphinxstylestrong{Next} button.
\begin{enumerate}
\setcounter{enumi}{1}
\item {} 
This is a search field to search the product which is issued just type the name of the product and it will show below the search field.

\item {} 
Here you can set the date for issued material and it will filter the list of material issued.

\item {} 
This is \sphinxstylestrong{Next} and \sphinxstylestrong{Previous} button to see the long list of material issued.

\item {} 
Click on \sphinxstylestrong{Material Issue} to download the PDF.

\end{enumerate}

\begin{figure}[htbp]
\centering

\noindent\sphinxincludegraphics{{materialissuedownload}.png}
\end{figure}

which will be in above format.
\begin{enumerate}
\setcounter{enumi}{5}
\item {} 
Click ob \sphinxstylestrong{Cancel} button of to cancel that project from the list of material issued.

\item {} 
Click on \sphinxstylestrong{Delivery Challan} button to \sphinxstyleemphasis{generate, edit and download} delivery challan.

\end{enumerate}

\begin{figure}[htbp]
\centering

\noindent\sphinxincludegraphics{{crtdeliverychallan}.png}
\end{figure}

Here fill the challan form to create the Delivery Challan .
\begin{enumerate}
\item {} 
Here enter the \sphinxstylestrong{Challan Number} .

\item {} 
Here enter the \sphinxstylestrong{Customer Name} .

\item {} 
Here enter the \sphinxstylestrong{Customer GST Number} .

\item {} 
Here enter the \sphinxstylestrong{Customer Address} .

\item {} 
Set the Challan Date either by selecting through Date Picker (which is just beside the text area) or just fill the date in this text area in DD-MM-YYYY format.

\item {} 
Here enter the \sphinxstylestrong{Reference Number} .

\item {} 
Here enter the \sphinxstylestrong{Medium of Delivery} .

\item {} 
Here enter the \sphinxstylestrong{Approximate Value} .

\item {} 
Here enter the \sphinxstylestrong{Heading Details} .

\item {} 
Here enter the \sphinxstylestrong{Notes} for customer.

\item {} 
Click on \sphinxstylestrong{Save} button and Delivery challan will be created after that when you will click on \sphinxstyleemphasis{Delivery Challan} you will get.

\end{enumerate}

\begin{figure}[htbp]
\centering

\noindent\sphinxincludegraphics{{editchallan}.png}
\end{figure}

Here you can edit the saved delivery challan and save it again.
\begin{enumerate}
\item {} 
Here you can edit the \sphinxstylestrong{Challan Number} .

\item {} 
Here you can edit the \sphinxstylestrong{Customer Name} .

\item {} 
Here you can edit the \sphinxstylestrong{Customer GST Number} .

\item {} 
Here you can edit the \sphinxstylestrong{Customer Address} .

\item {} 
Here you can edit the the Challan Date either by selecting through Date Picker (which is just beside the text area) or just fill the date in this text area in DD-MM-YYYY format.

\item {} 
Here you can edit the \sphinxstylestrong{Reference Number} .

\item {} 
Here you can edit the \sphinxstylestrong{Medium of Delivery} .

\item {} 
Here you can edit the \sphinxstylestrong{Approximate Value} .

\item {} 
Here you can edit the \sphinxstylestrong{Heading Details} .

\item {} 
Here you can edit the \sphinxstylestrong{Notes} for customer.

\item {} 
Click on \sphinxstylestrong{Save} button and Delivery challan will be updated.

\item {} 
To Download the PDF of delivery challan Click on \sphinxstylestrong{Download DC} .

\end{enumerate}

\begin{figure}[htbp]
\centering

\noindent\sphinxincludegraphics{{downloadedchallan}.png}
\end{figure}

which will be in above format.


\chapter{Vendor}
\label{\detokenize{vendor:vendor}}\label{\detokenize{vendor::doc}}
\begin{figure}[htbp]
\centering

\noindent\sphinxincludegraphics{{vendor}.png}
\end{figure}

As you can see above this is vendor tab. where you can create edit delete and search vendor details.
\begin{enumerate}
\item {} 
If you are want to create new vendor click on \sphinxstylestrong{New} and below tab will be open.

\end{enumerate}

\begin{figure}[htbp]
\centering

\noindent\sphinxincludegraphics{{crtnewvendor}.png}
\end{figure}

Now let’s start creating new vendor.
\begin{quote}
\begin{quote}
\begin{enumerate}
\item {} 
Here fill the \sphinxstylestrong{Name} of vendor.

\item {} 
Here enter the name of \sphinxstylestrong{Responsible Person} of that vendor.

\item {} 
Here fill \sphinxstylestrong{Mobile Number} .

\item {} 
Here enter the \sphinxstylestrong{Email id} of vendor.

\item {} 
Set the \sphinxstylestrong{GST} .

\item {} 
From here on wards keep feeling address of vendor like \sphinxstylestrong{Street} .

\item {} 
City  8. Pin code  9. State and  10. Country.

\end{enumerate}
\begin{enumerate}
\setcounter{enumi}{10}
\item {} 
Here you have two choice if you had filled wrong data about vendor click on \sphinxstylestrong{Reset} else click on \sphinxstylestrong{Save} to create new vendor. Now you will see one new vendor in your vendor list.

\end{enumerate}
\end{quote}
\begin{enumerate}
\setcounter{enumi}{1}
\item {} 
You can browse \sphinxstylestrong{Vendor} either in the appeared \sphinxstyleemphasis{list} or you can \sphinxstylestrong{Search} vendor in \sphinxstyleemphasis{search} bar.

\item {} 
By clicking on \sphinxstylestrong{vendor profile} you can \sphinxstyleemphasis{edit} vendor details.

\end{enumerate}
\end{quote}

\begin{figure}[htbp]
\centering

\noindent\sphinxincludegraphics{{editvendor}.png}
\end{figure}

The above tab is similar to \sphinxstyleemphasis{NEW} tab only differences are there is no \sphinxstyleemphasis{Reset} button and this tab fields are already filled so, you have to just edit some information according to your requirement and click on \sphinxstylestrong{Save} button so that given information will be updated.
\begin{enumerate}
\setcounter{enumi}{3}
\item {} 
By Clicking on \sphinxstylestrong{Delete} button of vendor you can delete that vendor from your vendors list.

\end{enumerate}


\chapter{Invoice}
\label{\detokenize{setting::doc}}\label{\detokenize{setting:invoice}}
\begin{figure}[htbp]
\centering

\noindent\sphinxincludegraphics{{invoice}.png}
\end{figure}

Here you can see all the invoices and their details and can manage as well.
\begin{enumerate}
\item {} 
Click on \sphinxstylestrong{New} button to create a new invoice

\end{enumerate}

\begin{figure}[htbp]
\centering

\noindent\sphinxincludegraphics{{crtinvoice}.png}
\end{figure}

In the above tab you have to fill below details for invoice.
\begin{quote}
\begin{enumerate}
\item {} 
Invoice Number.

\item {} 
Customer PO Reference.

\item {} 
Transporter Name from here on wards fill the Billing Details.

\item {} 
Here the name of billing person.

\item {} 
Here enter the billing address.

\item {} 
Here Enter the GSTN (GST Number).

\item {} 
State

\item {} 
Code (Pin Code)

\item {} 
Click on this check box and make it checked if Billing and shipping address is same otherwise fill that manually.

\item {} 
Invoice Date here

\item {} 
Insurance Number

\item {} 
LR(Lorry Receipt) Number.

\item {} 
Here enter the payment terms.

\item {} 
click on \sphinxstylestrong{+} icon to add new row to fill the project details in this table and it will show to several details and total amount on invoice.

\item {} 
Click on \sphinxstylestrong{Save} button to save the invoice.

\end{enumerate}
\begin{enumerate}
\setcounter{enumi}{1}
\item {} 
This is the \sphinxstylestrong{search} field to search the invoice by Invoice number.

\item {} 
Click on invoice number to check the invoice.

\end{enumerate}
\end{quote}

\begin{figure}[htbp]
\centering

\noindent\sphinxincludegraphics{{downloadinvoice}.png}
\end{figure}

In the above tab you can see the invoice.
\begin{quote}
\begin{enumerate}
\item {} 
Here you are seeing the Invoice related details.

\item {} 
IF you are looking for printed invoice then click on \sphinxstylestrong{Download Invoice} button it will generate the PDF of invoice you can print it.

\end{enumerate}
\begin{enumerate}
\setcounter{enumi}{3}
\item {} 
Click on Pencil button to Edit the invoice.

\end{enumerate}
\end{quote}

\begin{figure}[htbp]
\centering

\noindent\sphinxincludegraphics{{editinvice}.png}
\end{figure}

Editing invoice is similar to creating invoice. only difference is here details are already given you just have to change it.
\begin{quote}
\begin{enumerate}
\item {} 
Here is invoice details which you can edit according to your requirement.

\item {} 
Here you can delete the product from invoice by clicking on \sphinxstylestrong{delete} icon.

\item {} 
If you want to add some product in invoice click on \sphinxstylestrong{+} and fill the required details.

\item {} 
And in the last click on \sphinxstylestrong{save} button to save the invoice.

\end{enumerate}
\begin{enumerate}
\setcounter{enumi}{4}
\item {} 
Here you can delete the invoice by clicking on their \sphinxstylestrong{Delete} icon.

\end{enumerate}
\end{quote}


\chapter{Delivery Challan}
\label{\detokenize{setting:delivery-challan}}
\begin{figure}[htbp]
\centering

\noindent\sphinxincludegraphics{{deliverychallan}.png}
\end{figure}

As you click on Delivery Challan the above portal will open, here you can see the list of Delivery challan.
\begin{enumerate}
\item {} 
This is a \sphinxstylestrong{search} field to search the delivery challan by their \sphinxstylestrong{Challan Number}.

\item {} 
By clicking on that challan you can check their details which will look like

\end{enumerate}

\begin{figure}[htbp]
\centering

\noindent\sphinxincludegraphics{{downloadDC}.png}
\end{figure}

By clicking on \sphinxstylestrong{Download DC} button you will be able to download the Delivery challan in PDF format.


\chapter{Stock report}
\label{\detokenize{setting:stock-report}}
\begin{figure}[htbp]
\centering

\noindent\sphinxincludegraphics{{stockreport}.png}
\end{figure}

Above is the view of \sphinxstylestrong{Stock Report} .
\begin{enumerate}
\item {} 
This is a search field here you can search the Stock report by their name.

\item {} 
Click on \sphinxstylestrong{Download} icon to see the stock report for that specific date and name.

\item {} 
Click on \sphinxstylestrong{New} button to create a new stock report.

\end{enumerate}

\begin{figure}[htbp]
\centering

\noindent\sphinxincludegraphics{{stockreport1}.png}
\end{figure}

In the above form
\begin{enumerate}
\item {} 
Here enter (search and select) the \sphinxstylestrong{Product Number} .

\item {} 
Fill the \sphinxstylestrong{Quantity} .

\item {} 
And click on \sphinxstylestrong{Add} button to add the product in stock.

\item {} 
Click on delete button to \sphinxstylestrong{Delete} product from stock.

\item {} 
Click on \sphinxstylestrong{Save} button and the product will be added in stock report.

\item {} 
To check or share the stock report you can download by clicking on \sphinxstylestrong{Download} button. which will look like

\end{enumerate}

\begin{figure}[htbp]
\centering

\noindent\sphinxincludegraphics{{stockreportpdf}.png}
\end{figure}



\renewcommand{\indexname}{Index}
\printindex
\end{document}